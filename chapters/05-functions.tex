\chapter{Functions}

\section{Function definition}
A function is defined as a sequence of one or more \textbf{function
clauses}. The function name is an atom.

\vspace*{4pt}
\begin{erlang}
Func(Pattern11,...,Pattern1N) [when GuardSeq1] -> Body1;
    ...;
    ...;
Func(PatternK1,...,PatternKN) [when GuardSeqK] -> BodyK.

\end{erlang}
\vspace*{4pt}

The function clauses are separated by semicolons (\texttt{;}) and
terminated by full stop (\texttt{.}). A function clause consists of a
\textbf{clause head} and a \textbf{clause body} separated by an arrow
(\texttt{->}). A clause head consists of the function name (an atom),
arguments within parentheses and an optional guard sequence beginning
with the keyword \texttt{when}.  Each argument is a pattern.  A clause
body consists of a sequence of expressions separated by commas
(\texttt{,}).

\vspace*{4pt}
\begin{erlang}
Expr1,
...,
ExprM
\end{erlang}
\vspace*{4pt}

The number of arguments \texttt{N} is the \textbf{arity} of the
function. A function is uniquely defined by the module name, function name
and arity. Two different functions in the same module with different
arities may have the same name. A function \texttt{Func} in \texttt{Module}
with arity \texttt{N} is often denoted as \texttt{Module:Func/N}.

\vspace*{4pt}
\begin{erlang}
-module(mathStuff).
-export([area/1]).

area({square, Side}) -> Side * Side;
area({circle, Radius}) -> math:pi() * Radius * Radius;
area({triangle, A, B, C}) ->
    S = (A + B + C)/2,
    math:sqrt(S*(S-A)*(S-B)*(S-C)).
\end{erlang}

% push this section onto the next page (orphan line)
\newpage
\section{Function calls}
A function is called using:

\begin{erlang}
[Module:]Func(Expr1, ..., ExprN)
\end{erlang}

\texttt{Module} evaluates to a module name and \texttt{Func} to
a function name or a \textit{fun}. When calling a function in another
module, the module name must be provided and the function must be
exported. This is referred to as a \textbf{fully qualified function name}.

\begin{erlang}
lists:keysearch(Name, 1, List)
\end{erlang}

The module name can be omitted if \texttt{Func} evaluates to the name
of a local function, an imported function, or an auto-imported
BIF.  In such cases, the function is called using an \textbf{implicitly qualified function name}.

Before calling a function the arguments \texttt{ExprX} are
evaluated.  If the function cannot be found, an \texttt{undef} run-time
error will occur. Next the function clauses are scanned sequentially
until a clause is found such that the patterns in the clause head can
be successfully matched against the given arguments and that the guard
sequence, if any, is true. If no such clause can be found, a
\texttt{function\_clause} run-time error will occur.

If a matching clause is found, the corresponding clause body is evaluated,
i.e.~the expressions in the body are evaluated sequentially and the
value of the last expression is returned.

The fully qualified function name must be used when calling a function
with the same name as a BIF (built-in function, see section
\ref{functions:bifs}). The compiler does not allow defining a function
with the same name as an imported function. When calling a local
function, there is a difference between using the implicitly or fully
qualified function name, as the latter always refers to the latest
version of the module (see chapter \ref{code}).


\section{Expressions}
\label{functions:expressions}
An \textbf{expression} is either a term or the invocation of an
operator, for example:

\begin{erlang}
Term
op Expr
Expr1 op Expr2
(Expr)
begin
   Expr1,
   ...,
   ExprM            % no comma (,) before end
end

\end{erlang}

There are both \textit{unary} and \textit{binary} operators. The
simplest form of expression is a term, i.e.~an \textit{integer},
\textit{float}, \textit{atom}, \textit{string}, \textit{list} or
\textit{tuple} and the return value is the term itself. An expression
may contain \textit{macro} or \textit{record} expressions which will
expanded at compile time.

Parenthesised expressions are useful to override operator precedence (see section \ref{functions:expressions:precedence}):

\begin{erlang}
1 + 2 * 3           % 7
(1 + 2) * 3         % 9
\end{erlang}

Block expressions within \texttt{begin...end} can be used to group a
sequence of expressions and the return value is the value of the last
expression \texttt{ExprM}.

All subexpressions are evaluated before the expression itself is
evaluated, but the order in which subexpressions are evaluated is undefined.

Most operators can only be applied to arguments of a certain type. For
example, arithmetic operators can only be applied to integers or
floats. An argument of the wrong type will cause a \texttt{badarg}
run-time error.


\subsection{Term comparisons}
\begin{erlang}
Expr1 op Expr2
\end{erlang}

A \textbf{term comparison} returns a \textit{boolean} value,
in the form of atoms \texttt{true} or \texttt{false}.

\begin{center}
\begin{tabular}{|>{\raggedright}p{40pt}|>{\raggedright}p{105pt}|>{\raggedright}p{26pt}|>{\raggedright}p{124pt}|}
\hline
\multicolumn{4}{|p{297pt}|}{Comparison operators}\tabularnewline
\hline
\texttt{==} & Equal to & \texttt{=<} & Less than or equal to\tabularnewline
\hline
\texttt{/=} & Not equal to & \texttt{<} & Less than\tabularnewline
\hline
\texttt{=:=} & Exactly equal to & \texttt{>}= & Greater than or equal to\tabularnewline
\hline
\texttt{=/=} & Exactly not equal to & \texttt{>} & Greater than\tabularnewline
\hline
\end{tabular}
\end{center}

\begin{erlang}
1==1.0              % true
1=:=1.0             % false
1 > a               % false
\end{erlang}

The arguments may be of different data types. The following order is
defined:

\texttt{number < atom < reference < fun < port < pid < tuple < list < binary}

Lists are compared element by element. Tuples are ordered by size, two
tuples with the same size are compared element by element. When
comparing an integer and a float, the integer is first converted to a
float. In the case of \texttt{=:=} or \texttt{=/=} there is no type conversion.


\subsection{Arithmetic expressions}

\begin{erlang}
op Expr
Expr1 op Expr2
\end{erlang}

An \textbf{arithmetic expression} returns the result after applying
the operator.

\begin{center}
\begin{tabular}{|>{\raggedright}p{35pt}|>{\raggedright}p{145pt}|>{\raggedright}p{128pt}|}
\hline
\multicolumn{3}{|p{309pt}|}{Arithmetic operators}\tabularnewline
\hline
\texttt{+}  & Unary +  & \texttt{Integer \textbar{} Float} \tabularnewline
\hline
\texttt{-}  & Unary -  & \texttt{Integer \textbar{} Float}\tabularnewline
\hline
\texttt{+}  & Addition & \texttt{Integer} \textbar{} \texttt{Float}\tabularnewline
\hline
\texttt{-}  & Subtraction & \texttt{Integer} \textbar{} \texttt{Float}\tabularnewline
\hline
\texttt{*}  & Multiplication & \texttt{Integer} \textbar{} \texttt{Float}\tabularnewline
\hline
\texttt{/}  & Floating point division  & \texttt{Integer} \textbar{} \texttt{Float}\tabularnewline
\hline
\texttt{bnot}  & Unary bitwise not  & \texttt{Integer} \tabularnewline
\hline
\texttt{div}  & Integer division  & \texttt{Integer}\tabularnewline
\hline
\texttt{rem}  & Integer remainder of X/Y  & \texttt{Integer} \tabularnewline
\hline
\texttt{band}  & Bitwise and  & \texttt{Integer}\tabularnewline
\hline
\texttt{bor}  & Bitwise or  & \texttt{Integer} \tabularnewline
\hline
\texttt{bxor}  & Arithmetic bitwise xor  & \texttt{Integer}\tabularnewline
\hline
\texttt{bsl}  & Arithmetic bitshift left  & \texttt{Integer} \tabularnewline
\hline
\texttt{bsr}  & Bitshift right  & \texttt{Integer}\tabularnewline
\hline
\end{tabular}
\end{center}

\begin{erlang}
+1                  % 1
4/2                 % 2.00000
5 div 2             % 2
5 rem 2             % 1
2#10 band 2#01      % 0
2#10 bor 2#01       % 3
\end{erlang}


\subsection{Boolean expressions}

\begin{erlang}
op Expr
Expr1 op Expr2
\end{erlang}

A \textbf{boolean expression} returns the value \texttt{true} or
\texttt{false} after applying the operator.

\begin{center}
\begin{tabular}{|>{\raggedright}p{79pt}|>{\raggedright}p{241pt}|}
\hline
\multicolumn{2}{|p{321pt}|}{Boolean operators}\tabularnewline
\hline
\texttt{not}  & Unary logical not \tabularnewline
\hline
\texttt{and}  & Logical and \tabularnewline
\hline
\texttt{or}  & Logical or \tabularnewline
\hline
\texttt{xor}  & Logical exclusive or\tabularnewline
\hline
\end{tabular}
\end{center}

\begin{erlang}
not true            % false
true and false      % false
true xor false      % true
\end{erlang}


\subsection{Short-circuit boolean expressions}

\begin{erlang}
Expr1 orelse Expr2
Expr1 andalso Expr2
\end{erlang}

These are boolean expressions where \texttt{Expr2} is evaluated only
if necessary. In an \texttt{orelse} expression \texttt{Expr2} will be
evaluated only if \texttt{Expr1} evaluates to false. In an
\texttt{andalso} expression \texttt{Expr2} will be evaluated only if
\texttt{Expr1} evaluates to true.

\begin{erlang}
if A >= 0 andalso math:sqrt(A) > B -> ...

if is_list(L) andalso length(L) == 1 -> ...
\end{erlang}


\subsection{Operator precedences}
\label{functions:expressions:precedence}
In an expression consisting of subexpressions the operators will be
applied according to a defined \textbf{operator precedence} order:

\begin{center}
\begin{tabular}{|>{\raggedright}p{221pt}|>{\raggedright}p{99pt}|}
\hline
\multicolumn{2}{|p{321pt}|}{Operator precedence (from high to low)}\tabularnewline
\hline
\texttt{:} ~ &  \tabularnewline
\hline
\texttt{\#} ~ &  \tabularnewline
\hline
\texttt{Unary + - bnot not ~} &  \tabularnewline
\hline
\texttt{/ * div rem band and}  & Left associative \tabularnewline
\hline
\texttt{+ - bor bxor bsl bsr or xor} & Left associative \tabularnewline
\hline
\texttt{++ -{}-}  & Right associative \tabularnewline
\hline
\texttt{== /= =< < >= > =:= =/=} & \tabularnewline
\hline
\texttt{andalso}  &  \tabularnewline
\hline
\texttt{orelse} &  \tabularnewline
\hline
\texttt{= !}  & Right associative \tabularnewline
\hline
\texttt{catch ~} &  \tabularnewline
\hline
\end{tabular}
\end{center}

The operator with the highest priority is evaluated first. Operators
with the same priority are evaluated according to their
\textbf{associativity}.  The left associative arithmetic operators
are evaluated left to right:

\texttt{6 + 5 * 4 - 3 / 2  \resultingin  6 + 20 - 1.5  \resultingin  26 - 1.5  \resultingin  24.5}


\section{Compound expressions}


\subsection{If}

\begin{erlang}
if
    GuardSeq1 ->
        Body1;
    ...;
    GuardSeqN ->
        BodyN                   % Note no semicolon (;) before end
end
\end{erlang}

The branches of an \texttt{if} expression are scanned sequentially
until a guard sequence \texttt{GuardSeq} which evaluates to
\texttt{true} is found.  The corresponding \texttt{Body} (sequence
of expressions separated by commas) is then evaluated.  The return value of
\texttt{Body} is the return value of the \texttt{if} expression.

If no guard sequence is true, an \texttt{if\_clause} run-time error
will occur. If necessary, the guard expression \texttt{true} can be used in the
last branch, as that guard sequence is always true (known as a ``catch
all'').

\begin{erlang}
is_greater_than(X, Y) ->
    if
        X>Y ->
            true;
        true ->                 % works as an 'else' branch
            false
    end
\end{erlang}

It should be noted that pattern matching in function clauses can be used to replace \texttt{if} cases (most of the time).
Overuse of \texttt{if} sentences withing function bodies is considered a bad Erlang practice.

\subsection{Case}

Case expressions provide for inline pattern matching, similar to the way in which function clauses are matched.

\begin{erlang}
case Expr of
    Pattern1 [when GuardSeq1] ->
        Body1;
        ...;
    PatternN [when GuardSeqN] ->
        BodyN                   % Note no semicolon (;) before end
end
\end{erlang}

The expression \texttt{Expr} is evaluated and the patterns
\texttt{Pattern1}...\texttt{PatternN} are sequentially matched against the result. If a
match succeeds and the optional guard sequence \texttt{GuardSeqX} is
\texttt{true}, then the corresponding \texttt{BodyX} is evaluated. The return value
of \texttt{BodyX} is the return value of the case expression.

If there is no matching pattern with a true guard sequence, a
\texttt{case\_clause} run-time error will occur.

\begin{erlang}
is_valid_signal(Signal) ->
    case Signal of
        {signal, _What, _From, _To} ->
            true;
        {signal, _What, _To} ->
            true;
        _Else ->                % 'catch all'
            false
    end.
\end{erlang}


\subsection{List comprehensions}
List comprehensions are analogous to the \texttt{setof} and
\texttt{findall} predicates in Prolog.

\begin{erlang}
[Expr || Qualifier1,...,QualifierN]
\end{erlang}

\texttt{Expr} is an arbitrary expression, and each \texttt{QualifierX}
is either a \textbf{generator} or a \textbf{filter}. A generator is
written as:

\begin{erlang}
Pattern <- ListExpr
\end{erlang}

where \texttt{ListExpr} must be an expression which evaluates to a
list of terms. A filter is an expression which evaluates to
\texttt{true} or \texttt{false}. Variables in list generator
expressions \textit{shadow} variables in the function clause
surrounding the list comprehension.

The qualifiers are evaluated from left to right, the generators
creating values and the filters constraining them. The list
comprehension then returns a list where the elements are the result of
evaluating \texttt{Expr} for each combination of the resulting values.

\begin{minted}{console}
> [{X, Y} || X <- [1,2,3,4,5,6], X > 4, Y <- [a,b,c]].
[{5,a},{5,b},{5,c},{6,a},{6,b},{6,c}]
\end{minted}


\section{Guard sequences}
A \textbf{guard sequence} is a set of \textbf{guards} separated by
semicolons (\texttt{;}). The guard sequence is \texttt{true} if at
least one of the guards is \texttt{true}.

\begin{erlang}
Guard1; ...; GuardK
\end{erlang}

A \textbf{guard} is a set of \textbf{guard expressions} separated by
commas (\texttt{,}). The guard is \texttt{true} if all guard
expressions evaluate to \texttt{true}.

\begin{erlang}
GuardExpr1, ..., GuardExprN
\end{erlang}

The permitted \textbf{guard expressions} (sometimes called guard
tests) are a subset of valid Erlang expressions, since the
evaluation of a guard expression must be guaranteed to be free of side-effects.

\begin{center}
\begin{tabular}{|>{\raggedright}p{154pt}|>{\raggedright}p{166pt}|}
\hline
\multicolumn{2}{|p{321pt}|}{Valid guard expressions:}\tabularnewline
\hline
\multicolumn{2}{|p{321pt}|}{The atom \texttt{true};}\tabularnewline
\hline
\multicolumn{2}{|p{321pt}|}{Other constants (terms and bound variables), are all regarded
as \texttt{false};}\tabularnewline
\hline
\multicolumn{2}{|p{321pt}|}{Term comparisons;}\tabularnewline
\hline
\multicolumn{2}{|p{321pt}|}{Arithmetic and boolean expressions;}\tabularnewline
\hline
\multicolumn{2}{|p{321pt}|}{Calls to the BIFs specified below.}\tabularnewline
\hline
Type test BIFs & Other BIFs allowed in guards:\tabularnewline
\hline
\texttt{is\_atom/1} & \texttt{abs(Integer} \textbar{} \texttt{Float)}\tabularnewline
\hline
\texttt{is\_constant/1} & \texttt{float(Term)}\tabularnewline
\hline
\texttt{is\_integer/1} & \texttt{trunc(Integer} \textbar{} \texttt{Float)}\tabularnewline
\hline
\texttt{is\_float/1} & \texttt{round(Integer} \textbar{} \texttt{Float)}\tabularnewline
\hline
\texttt{is\_number/1} & \texttt{size(Tuple} \textbar{} \texttt{Binary)}\tabularnewline
\hline
\texttt{is\_reference/1} & \texttt{element(N, Tuple)}\tabularnewline
\hline
\texttt{is\_port/1} & \texttt{hd(List)}\tabularnewline
\hline
\texttt{is\_pid/1} & \texttt{tl(List)}\tabularnewline
\hline
\texttt{is\_function/1} & \texttt{length(List)}\tabularnewline
\hline
\texttt{is\_tuple/1} & \texttt{self()}\tabularnewline
\hline
\texttt{is\_record/2} The 2\textsuperscript{nd} argument is \linebreak{}
the record name & \texttt{node(})\tabularnewline
\hline
\texttt{is\_list/1} & \texttt{node(Pid} \textbar{} \texttt{Ref} \textbar \texttt{Port)}\tabularnewline
\hline
\texttt{is\_binary/1} & \tabularnewline
\hline
\end{tabular}
\end{center}

A small example:

\begin{erlang}
fact(N) when N>0 ->             % first clause head
    N * fact(N-1);              % first clause body
fact(0) ->                      % second clause head
    1.                          % second clause body
\end{erlang}


\section{Tail recursion}
If the last expression of a function body is a function call, a
\textbf{tail recursive} call is performed in such a way that no system
resources (like the call stack) are consumed. This means that an
infinite loop like a server can be programmed provided it only uses
tail recursive calls.

The function \texttt{fact/1} above could be rewritten using tail
recursion in the following manner:

 \begin{erlang}
fact(N) when N>1 -> fact(N, N-1);
fact(N) when N==1; N==0 -> 1.

fact(F,0) -> F;                 % The variable F is used as an accumulator
fact(F,N) -> fact(F*N, N-1).
\end{erlang}


\section{Funs}
\label{functions:funs}
A \textbf{fun} defines a \textit{functional object}. Funs make it
possible to pass an entire function, not just the function name, as an
argument. A `fun' expression begins with the keyword \texttt{fun} and
ends with the keyword \texttt{end} instead of a full stop
(\texttt{.}).  Between these should be a regular function
declaration, except that no function name is specified.

\begin{erlang}
fun
    (Pattern11,...,Pattern1N) [when GuardSeq1] ->
        Body1;
        ...;
    (PatternK1,...,PatternKN) [when GuardSeqK] ->
        BodyK
end
\end{erlang}

Variables in a \texttt{fun} head \textit{shadow} variables in the function
clause surrounding the \texttt{fun} but variables bound in a \texttt{fun} body are local
to the body.  The return value of the expression is the resulting function. The expression
\texttt{fun Name/N is} equivalent to:

\begin{erlang}
fun (Arg1,...,ArgN) -> Name(Arg1,...,ArgN) end
\end{erlang}

The expression \texttt{fun Module:Func/Arity} is also allowed, provided that \texttt{Func} is exported
from \texttt{Module}.

\begin{erlang}
Fun1 = fun (X) -> X+1 end.
Fun1(2)         % 3

Fun2 = fun (X) when X>=1000 -> big; (X) -> small end.
Fun2(2000)      % big
\end{erlang}

Anonymous \texttt{funs}: When a \texttt{fun} is anonymous, i.e.~there is no function name in the definition of the \texttt{fun}, the definition of a recursive \texttt{fun} has to be done in two steps. This example shows how to define an anonymous \texttt{fun} \texttt{sum(List)} (see section \ref{datatypes:list}) as an anonymous \texttt{fun}.

\begin{erlang}
Sum1 = fun ([], _Foo) -> 0;([H|T], Foo) -> H + Foo(T, Foo) end.
Sum = fun (List) -> Sum1(List, Sum1) end.
Sum([1,2,3,4,5])    % 15
\end{erlang}

The definition of \texttt{Sum} is done in a way such that it takes \textit{itself} as a parameter, matched to \texttt{\_Foo} (empty list) or \texttt{Foo},
which it then calls recursively.  The definition of \texttt{Sum} calls \texttt{Sum1}, also passing \texttt{Sum1} as a parameter.


Names in \texttt{funs}: In Erlang you can use a name inside a \texttt{fun} before the name has been
defined. The syntax of \texttt{funs with names} allows a variable name to be
consistently present before each argument list. This allows \texttt{funs} to be
recursive in one steps. This example shows how to define the function
\texttt{sum(List)} (see section \ref{datatypes:list}) as a \texttt{funs with names}.

\begin{erlang}
Sum = fun Sum([])-> 0;Sum([H|T]) -> H + Sum(T) end.
Sum([1,2,3,4,5])    % 15
\end{erlang}


\section{BIFs --- Built-in functions}
\label{functions:bifs}
The \textbf{built-in functions}, BIFs, are implemented in the C code of
the runtime system and do things that are difficult or impossible to
implement in Erlang. Most of the built-in functions belong to the
module \texttt{erlang} but there are also built-in functions that belong
to other modules like \texttt{lists} and \texttt{ets}. The most
commonly used BIFs belonging to the module \texttt{erlang} are
\textbf{auto-imported}, i.e.~they do not need to be prefixed with the
module name.

\begin{center}
\begin{tabular}{|>{\raggedright}p{103pt}|>{\raggedright}p{217pt}|}
\hline
\multicolumn{2}{|p{321pt}|}{Some useful BIFs}\tabularnewline
\hline
\texttt{date()} & Returns today's date as \texttt{\{Year, Month, Day\}}\tabularnewline
\hline
\texttt{now()} & Returns current time in microseconds. System dependent\tabularnewline
\hline
\texttt{time()} & Returns current time as \texttt{\{Hour, Minute, Second\}} System dependent\tabularnewline
\hline
\texttt{halt()} & Stops the Erlang system\tabularnewline
\hline
\texttt{processes()} & Returns a list of all processes currently known to the system\tabularnewline
\hline
\texttt{process\_info(Pid)} & Returns a dictionary containing information about \texttt{Pid}\tabularnewline
\hline
\texttt{Module:module\_info()} & Returns a dictionary containing information about the code
in Module\tabularnewline
\hline
\end{tabular}
\end{center}

A \textbf{dictionary} is a list of \texttt{\{Key, Value\} terms (see
also section \ref{processes:dicts}).}

\begin{erlang}
size({a, b, c})             % 3
atom_to_list('Erlang')      % "Erlang"
date()                      % {2013,5,27}
time()                      % {01,27,42}
\end{erlang}
