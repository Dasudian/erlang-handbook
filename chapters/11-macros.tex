\chapter{Macros}
\label{macros}

\section{Defining and using macros}

\begin{erlang}
-define(Const, Replacement).
-define(Func(Var1, ..., VarN), Replacement).
\end{erlang}

A \textbf{macro} must be defined before it is used but a macro
definition may be placed anywhere among the attributes and function
declarations of a module. If a macro is used in several modules it is
advisable to put the macro definition in an include file. A macro is
used as follows:

\begin{erlang}
?Const
?Func(Arg1,...,ArgN)
\end{erlang}

Macros are expanded during compilation. A macro reference
\texttt{?Const} is replaced by \texttt{Replacement} like this:

\begin{erlang}
-define(TIMEOUT, 200).
...
call(Request) ->
    server:call(refserver, Request, ?TIMEOUT).
\end{erlang}

is expanded to:

\begin{erlang}
call(Request) ->
    server:call(refserver, Request, 200).
\end{erlang}

A macro reference \texttt{?Func(Arg1, ..., ArgN)} will be replaced by
\texttt{Replacement}, where all occurences of a variable \texttt{VarX}
from the macro definition are replaced by the corresponding argument
\texttt{ArgX}.

\begin{erlang}
-define(MACRO1(X, Y), {a, X, b, Y}).
...
bar(X) ->
    ?MACRO1(a, b),
    ?MACRO1(X, 123).
\end{erlang}

will be expanded to:

\begin{erlang}
bar(X) ->
    {a, a, b, b},
    {a, X, b, 123}.
\end{erlang}

To view the result of macro expansion, a module can be compiled with
the \texttt{`P'} option:

\begin{erlang}
compile:file(File, ['P']).
\end{erlang}

This produces a listing of the parsed code after preprocessing and
parse transforms, in the file \texttt{File.P}.


\section{Predefined macros}

\begin{center}
\begin{tabular}{|>{\raggedright}p{103pt}|>{\raggedright}p{223pt}|}
\hline
\multicolumn{2}{|p{326pt}|}{P{\large{}redefined} macros}\tabularnewline
\hline
\texttt{?MODULE} & The name of the current module\tabularnewline
\hline
\texttt{?MODULE\_STRING} & The name of the current module, as a string\tabularnewline
\hline
\texttt{?FILE} & The file name of the current module\tabularnewline
\hline
\texttt{?LINE} & The current line number\tabularnewline
\hline
\texttt{?MACHINE} & The machine name, 'BEAM'\tabularnewline
\hline
\end{tabular}
\end{center}


\section{Flow Control in Macros}

\begin{erlang}
-undef(Macro).      % This inhibits the macro definition.

-ifdef(Macro).
    %% Lines that are evaluated if Macro was defined
-else.
    %% If the condition was false, these lines are evaluated instead.
-endif.
\end{erlang}

\texttt{ifndef(Macro)} can be used instead of \texttt{ifdef} and means
the opposite.

\begin{erlang}
-ifdef(debug).
-define(LOG(X), io:format("{~p,~p}:~p~n",[?MODULE,?LINE,X])).
-else.
-define(LOG(X), true).
-endif.
\end{erlang}

If \texttt{debug} is defined when the module is compiled,
\texttt{?LOG(Arg)} will expand to a call to \texttt{io:format/2} and
provide the user with some simple trace output.


\section{Stringifying Macro Arguments}
\texttt{??Arg}, where \texttt{Arg} is a macro argument expands to the
argument in the form of a string.

\begin{erlang}
-define(TESTCALL(Call), io:format("Call ~s: ~w~n", [??Call, Call])).

?TESTCALL(myfunction(1,2)),
?TESTCALL(you:function(2,1)),
\end{erlang}

results in:

\begin{erlang}
io:format("Call ~s: ~w~n", ["myfunction(1,2)", m:myfunction(1,2)]),
io:format("Call ~s: ~w~n", ["you:function(2,1)", you:function(2,1)]),
\end{erlang}

That is, a trace output with both the function called and the
resulting value.