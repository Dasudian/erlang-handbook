\chapter{Structure of an Erlang program}

\section{Module syntax}

An Erlang program is made up of \textbf{modules} where each module is
a text file with the extension \textbf{.erl}. For a small program like
a course example, all modules would reside in one directory. A module
consists of module attributes and function definitions.

\begin{erlang}
-module(demo).
-export([double/1]).

double(X) -> times(X, 2).

times(X, N) -> X * N.
\end{erlang}

The module \texttt{demo} consists of the function \texttt{times/2}
which is local to the module and the function \texttt{double/1} which
can be called from outside the module.

\texttt{demo:double(10)} $\Rightarrow$ \texttt{20}
(the arrow $\Rightarrow$ should be read as ``resulting in'')

\texttt{double/1} means the function double with \textit{one}
argument. A function \texttt{double/2} taking \textit{two} arguments
is regarded as a different function. The number of arguments is called
the \textbf{arity} of the function.


\section{Module attributes}
A \textbf{module attribute} defines a certain property of a module. A
module attribute consists of a \textbf{tag} and a \textbf{value}:

\texttt{-Tag(Value).}

\textbf{Tag} must be an atom, while \textbf{Value} must be a literal
term (see chapter \ref{datatypes}). Any module attribute can be specified. The
attributes are stored in the compiled code and can be retrieved by
calling the function \texttt{Module:module\_info(attributes).}

\subsection{Pre-defined module attributes}
Pre-defined module attributes must be placed before any function
declaration.

\begin{itemize}

	\item \begin{erlangim}
	-module(Module).
	\end{erlangim}
	This attribute is mandatory and must be specified first. It
        defines the name of the module. The name Module, an atom,
        should be the same as the file name without the extension erl.

	\item \begin{erlangim}
	-export([Func1/Arity1, ..., FuncN/ArityN]).
	\end{erlangim}
	 This attribute specifies which functions in the module that
         can be called from outside the module. Each function name
         FuncX is an atom and ArityX an integer.

	\item \begin{erlangim}
	-import(Module,[Func1/Arity1, ..., FuncN/ArityN]).
	\end{erlangim}
	This attribute defines a Module from which a list of functions
        are imported.

	\item \begin{erlangim}
	-import(demo, [double/1]).
	\end{erlangim}
	This means that it is possible to write double(10) instead of
        the longer demo:double(10) which can be practical if the
        function is used frequently.

	\item \begin{erlangim}
	-compile(Options).
	\end{erlangim}
	Compiler options.

	\item \begin{erlangim}
	-vsn(Vsn).
	\end{erlangim}
	Module version. If this attribute is not specified, the
        version defaults to the checksum of the module.

	\item \begin{erlangim}
	-behaviour(Behaviour).
	\end{erlangim}
	This attribute either specifies a user defined behaviour or
        one of the OTP standard behaviours \texttt{gen\_server},
        \texttt{gen\_fsm}, \texttt{gen\_event} or
        \texttt{supervisor}. The spelling behavior is also accepted.

\end{itemize}


\subsection{Macro and record definitions}

Records and macros are defined in the same way as module attributes

\begin{erlang}
-record(Record,Fields).

-define(Macro,Replacement).
\end{erlang}

Records and macro definitions are also allowed between functions, as
long as the definition comes before its first use. (About records see
section \ref{datatypes:record} and about macros see chapter \ref{macros})

\subsection{File inclusion}

File inclusion is specified in the same way as module attributes.

\begin{erlang}
-include(File).

-include_lib(File).
\end{erlang}

\texttt{File} is a string that represents a file name. Include files
are typically used for record and macro definitions that are shared by
several modules. By convention, the extension \texttt{.hrl} is used
for include files.

\begin{erlang}
-include("my_records.hrl").
-include("incdir/my_records.hrl").
-include("/home/user/proj/my_records.hrl").
\end{erlang}

File may start with a path component \texttt{\$Var} then the value of
the environment variable \texttt{Var} returned by
\texttt{os:getenv(Var)} is substituted for \texttt{\$Var}.

\begin{erlang}
-include("$PROJ_ROOT/my_records.hrl").
\end{erlang}
%%$ texmaker parser bug

\texttt{include\_lib} is similar to \texttt{include}, but then the
first path component is assumed to be the name of an application.

\begin{erlang}
-include_lib("kernel/include/file.hrl").
\end{erlang}

The code server uses \texttt{code:lib\_dir(kernel)} to find the
directory of the current (latest) version of \texttt{kernel}, and then
the subdirectory include is searched for the file \texttt{file.hrl}.


\section{Comments}
Comments may appear anywhere in a module except within strings and
quoted atoms.  A comment begins with the percentage character
(\texttt{\%}) and covers the rest of the line but not the
end-of-line. The terminating end-of-line has the effect of a blank.


\section{Character Set}
Erlang handles the full Latin-1 (ISO-8859-1) character set. Thus all
Latin-1 printable characters can be used and displayed without the
escape backslash. Atoms and variables can use all Latin-1 letters.

\begin{center}
\begin{tabular}{|>{\raggedright}p{52pt}|>{\raggedright}p{53pt}|>{\raggedright}p{103pt}|>{\raggedright}p{87pt}|}
\hline
\multicolumn{4}{|p{297pt}|}{Character classes}\tabularnewline
\hline
Octal & Decimal~ &   & Class\tabularnewline
\hline
40 -  57 & 32 - 47 &  ! \texttt{"} \# \$ \% \& ' / & Punctuation
characters\tabularnewline
\hline
60 -  71 & 48 - 57 & 0 - 9 & Decimal digits\tabularnewline
\hline
72 - 100 & 58 - 64 & : ; \texttt{<} = \texttt{>} @ & Punctuation characters\tabularnewline
\hline
101 - 132 &  65 - 90 & A - Z & Uppercase letters\tabularnewline
\hline
133 - 140 &  91 - 96 & [ \textbackslash{} ] \textasciicircum{} \_ ` & Punctuation
characters\tabularnewline
\hline
141 - 172 &  97 - 122 & a  -  z & Lowercase letters\tabularnewline
\hline
173 - 176 & 123 - 126 & \{ \textbar{} \} \textasciitilde{} & Punctuation characters\tabularnewline
\hline
200 - 237 & 128 - 159 ~ &   & Control characters \tabularnewline
\hline
240 - 277 & 160 - 191 & - ¿  & Punctuation characters \tabularnewline
\hline
300 - 326 & 192 - 214 & À - Ö  & Uppercase letters \tabularnewline
\hline
327  & 215 & ×  & Punctuation character \tabularnewline
\hline
330 - 336 & 216 - 222 & Ø - Þ  & Uppercase letters \tabularnewline
\hline
337 - 366 & 223 - 246 & ß - ö  & Lowercase letters \tabularnewline
\hline
367  & 247 & ÷  & Punctuation character \tabularnewline
\hline
370 - 377 & 248 - 255 & ø - ÿ  & Lowercase letters \tabularnewline
\hline
\end{tabular}
\end{center}

\section{Reserved words}

\vspace{12pt}

The following are reserved words in Erlang:

\begin{erlang}
after and andalso band begin bnot bor bsl bsr bxor case catch cond
div end fun if let not of or orelse receive rem try when xor
\end{erlang}
