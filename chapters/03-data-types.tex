\chapter{Data types (terms)}
\label{datatypes}

\section{Unary data types}

\subsection{Atoms}
\label{datatypes:atom}
An \textbf{atom} is a symbolic name, also known as a
\textit{literal}.  Atoms begin with a lower-case letter, and may contain alphanumeric characters, underscores (\texttt{\_}) or at-signs (\texttt{@}).
Alternatively atoms can be specified by enclosing them in single quotes (\texttt{'}), necessary when they start with an uppercase character or contain
characters other than underscores and at-signs.  For example:

\texttt{hello}

\texttt{phone\_number}

\texttt{'Monday'}

\texttt{'phone number'}

\texttt{'Anything inside quotes \textbackslash n\textbackslash 012'}
\hfill(see section \ref{datatypes:escapeseq})


\subsection{Booleans}
\label{datatypes:boolean}
There is no \textbf{boolean} data type in Erlang. The atoms \texttt{true} and
\texttt{false} are used instead.

\texttt{2 =< 3} \resultingin \texttt{true} \\
\texttt{true or false} \resultingin \texttt{true}


\subsection{Integers}
\label{datatypes:integer}
In addition to the normal way of writing \textbf{integers} Erlang
provides further notations. \texttt{\$Char} is the Latin-1 numeric value of the
character `\texttt{Char}' (that may be an escape sequence) and \texttt{Base\#Value} is an integer in base \texttt{Base}, which
must be an integer in the range $2..36$.

\texttt{42} \resultingin \texttt{42} \\
\$A  \resultingin \texttt{65} \\
\texttt{\$\textbackslash n} \resultingin \texttt{10}\hfill(see section \ref{datatypes:escapeseq}) \\
\texttt{2\#101} \resultingin \texttt{5} \\
\texttt{16\#1f} \resultingin \texttt{31}


\subsection{Floats}
\label{datatypes:float}
A \textbf{float} is a real number written \texttt{Num}[\texttt{eExp}]
where \texttt{Num} is a decimal number between 0.01 and 10000 and
\texttt{Exp} (optional) is a signed integer specifying the power-of-10 exponent.  For example:

\texttt{2.3e-3} \resultingin \texttt{2.30000e-3}\hfill
(corresponding to 2.3*10\textsuperscript{-3})


\subsection{References}
\label{datatypes:reference}
A \textbf{reference} is a term which is unique in an Erlang runtime
system, created by the built-in function \texttt{make\_ref/0}.  (For more information on built-in functions, or \textit{BIF}s, see section \ref{functions:bifs}.)


\subsection{Ports}
\label{datatypes:port}
A \textbf{port identifier} identifies a port (see chapter \ref{ports}).


\subsection{Pids}
\label{datatypes:pid}
A \textbf{process identifier}, \textit{pid}, identifies a process (see
chapter \ref{processes}).


\subsection{Funs}
\label{datatypes:fun}
A \textit{fun} identifies a \textbf{functional object} (see section
\ref{functions:funs}).


\section{Compound data types}

\subsection{Tuples}
\label{datatypes:tuple}
A \textbf{tuple} is a compound data type that holds a \textbf{fixed
number of terms} enclosed within curly braces.

\texttt{\{Term1,...,TermN\}}

Each \texttt{TermX} in the tuple is called an \textbf{element}. The
number of elements is called the \textbf{size} of the tuple.

\begin{center}
\begin{tabular}{|>{\raggedright}p{134pt}|>{\raggedright}p{186pt}|}
\hline
\multicolumn{2}{|p{321pt}|}{BIFs to manipulate tuples}\tabularnewline
\hline
\texttt{size(Tuple)} & Returns the size of \texttt{Tuple}\tabularnewline
\hline
\texttt{element(N,Tuple)} & Returns the \texttt{N}\textsuperscript{th} element in \texttt{Tuple}\tabularnewline
\hline
\texttt{setelement(N,Tuple,Expr)} & Returns a new tuple copied from \texttt{Tuple} except that the
\texttt{N}\textsuperscript{th} element is replaced by \texttt{Expr}\tabularnewline
\hline
\end{tabular}
\end{center}

\texttt{P = \{adam, 24, \{july, 29\}\}} \resultingin \texttt{P} is bound to \texttt{\{adam, 24, \{july, 29\}\}} \\
\texttt{element(1, P)} \resultingin \texttt{adam} \\
\texttt{element(3, P)} \resultingin \texttt{ \{july,29\}} \\
\texttt{P2 = setelement(2, P, 25)} \resultingin \texttt{P2} is bound to \texttt{\{adam, 25, \{july, 29\}\}} \\
\texttt{size(P)} \resultingin \texttt{3} \\
\texttt{size(\{\})} \resultingin \texttt{0} \\


\subsection{Records}
\label{datatypes:record}
A \textbf{record} is a \textit{named tuple} with named elements
called \textbf{fields}.  A record type is defined as a module attribute, for example:

\begin{erlang}
-record(Rec, {Field1 [= Value1],
              ...
              FieldN [= ValueN]}).
\end{erlang}

\texttt{Rec} and \texttt{Fields} are atoms and each \texttt{FieldX}
can be given an optional default \texttt{ValueX}. This definition may
be placed amongst the functions of a module, but only before it is used.  If a record type is
used by several modules it is advisable to put it in a separate file for inclusion.

A new record of type \texttt{Rec} is created using an expression like this:

\begin{erlang}
#Rec{Field1=Expr1, ..., FieldK=ExprK [, _=ExprL]}
\end{erlang}

The fields need not be in the same order as in the record
definition. Fields omitted will get their respective default
values. If the final clause is used, omitted fields will get the value
\texttt{ExprL}. Fields without default values and that are omitted
will have an undefined value.

The value of a field is retrieved using the expression
``\texttt{Variable\#Rec.Field}''.

\begin{erlang}
-module(employee).
-export([new/2]).
-record(person, {name, age, employed=erixon}).

new(Name, Age) -> #person{name=Name, age=Age}.
\end{erlang}

The function \texttt{employee:new/2} can be used in another module
which must also include the same record definition of \texttt{person}.

\texttt{\{P = employee:new(ernie,44)\}} \resultingin \texttt{\{person, ernie, 44,
erixon\}} \\
\texttt{P\#person.age} \resultingin \texttt{44} \\
\texttt{P\#person.employed} \resultingin \texttt{erixon}

When working with records in the Erlang shell, the functions \texttt{rd(RecordName, RecordDefinition)} and \texttt{rr(Module)} can be used to
define and load record definitions.  Refer to the \textit{Erlang Reference Manual} for more information.


\subsection{Lists}
\label{datatypes:list}
A \textbf{list} is a compound data type that holds a \textit{variable}
number of \textbf{terms} enclosed within square brackets.

\texttt{[Term1,...,TermN]}

Each term \texttt{TermX} in the list is called an
\textbf{element}.  The \textbf{length} of a list refers to the number of elements.  Common in functional programming,
the first element is called the \textbf{head} of the list
and the remainder (from the 2\textsuperscript{nd} element onwards) is called
the \textbf{tail} of the list.  Note that individual elements within a list do not have to have the same type, although it is common (and perhaps good)
practice to do so --- where mixed types are involved, \textbf{records} are more commonly used.

\begin{center}
\begin{tabular}{|>{\raggedright}p{90pt}|>{\raggedright}p{230pt}|}
\hline
\multicolumn{2}{|p{321pt}|}{BIFs to manipulate lists}\tabularnewline
\hline
\texttt{length(List)} & Returns the length of \texttt{List}\tabularnewline
\hline
\texttt{hd(List)} & Returns the 1\textsuperscript{st} (head) element of \texttt{List}\tabularnewline
\hline
\texttt{tl(List)} & Returns \texttt{List} with the 1\textsuperscript{st} element removed (tail)\tabularnewline
\hline
\end{tabular}
\end{center}

The vertical bar operator (\textbar{}) separates the leading elements of a list (one or more) from the remainder.  For example:

\texttt{[H | T]  = [1, 2, 3, 4, 5]} \resultingin \texttt{H=1} and \texttt{T=[2, 3, 4, 5]} \\
\texttt{[X, Y | Z] = [a, b, c, d, e]} \resultingin \texttt{X=a}, \texttt{Y=b} and \texttt{Z=[c, d, e]}

Implicitly a list will end with an empty list, i.e.~\texttt{[a, b]} is
the same as \texttt{[a, b | []]}.  A list looking like \texttt{[a, b | c]}
is \textbf{badly formed} and should be avoided (because the atom '\texttt{c}' is \textit{not} a list).
Lists lend themselves naturally to recursive functional programming.  For example, the following
function `\texttt{sum}' computes the sum of a list, and `\texttt{double}' multiplies each element in a list by 2, constructing and returning a new list
as it goes.

\begin{erlang}
sum([]) -> 0;
sum([H | T]) -> H + sum(T).

double([]) -> [];
double([H | T]) -> [H*2 | double(T)].
\end{erlang}

The above definitions introduce \textit{pattern matching}, described in chapter \ref{patterns}.  Patterns of this form are common in recursive programming,
explicitly providing a ``base case'' (for the empty list in these examples).

For working with lists, the operator \texttt{++} joins two lists together (appends the second argument to the first) and
returns the resulting list.  The operator \texttt{--} produces a list that
is a copy of its first argument, except that for each element in
the second argument, the first occurrence of this element (if any) in the resulting list is
removed.

\texttt{[1,2,3] ++ [4,5]} \resultingin \texttt{[1,2,3,4,5]}
\texttt{[1,2,3,2,1,2] -- [2,1,2]} \resultingin \texttt{[3,1,2]}

A collection of list processing functions can be found in the
\texttt{STDLIB} module \texttt{lists}.


\subsection{Strings}
\label{datatypes:string}
\textbf{Strings} are character strings enclosed within double quotes
but are, in fact, stored as lists of characters.

\texttt{"abcdefghi"} is the same as \texttt{[97,98,99,100,101,102,103,104,105]}

\texttt{""} is the same as \texttt{[]}

Two adjacent strings will be concatenated into one at compile-time and
do not incur any runtime overhead.

\texttt{"string" "42"} \resultingin \texttt{"string42"}


\subsection{Binaries}
\label{datatypes:binary}
A binary is a chunk of untyped memory by default a sequence of 8-bit
bytes.

\texttt{<}\texttt{<Elem1,...,ElemN>}\texttt{>}

Each \texttt{ElemX} is specified as \texttt{Value}[\texttt{:Size}][\texttt{/TypeSpecifierList}].

\begin{center}
\begin{tabular}{|>{\raggedright}p{73pt}|>{\raggedright}p{81pt}|>{\raggedright}p{147pt}|}
\hline
\multicolumn{3}{|p{297pt}|}{Element specification}\tabularnewline
\hline
\texttt{Value} & \texttt{Size} & \texttt{TypeSpecifierList}\tabularnewline
\hline
Should evaluate to an integer, float or binary & Should
evaluate to an integer & A sequence of optional type specifiers, in any order,
separated by hyphens (-)\tabularnewline
\hline
\end{tabular}
\end{center}

% page break at this point, so turned into two chunks.
\begin{center}
\begin{tabular}{|>{\raggedright}p{47pt}|>{\raggedright}p{115pt}|>{\raggedright}p{147pt}|}
\hline
\multicolumn{3}{|p{297pt}|}{Type specifiers}\tabularnewline
\hline
Type & \texttt{integer} \textbar{} \texttt{float} \textbar{} \texttt{binary} & Default
is \texttt{integer}\tabularnewline
\hline
Signedness & \texttt{signed} \textbar{} \texttt{unsigned} & Default is
\texttt{unsigned}\tabularnewline
\hline
Endianness & \texttt{big} \textbar{} \texttt{little} \textbar{} \texttt{native} & CPU
dependent. Default is \texttt{big}\tabularnewline
\hline
Unit & \texttt{unit:}\textit{IntegerLiteral} & Allowed range is $1..256$.
Default is 1 for integer and float, and 8 for binary\tabularnewline
\hline
\end{tabular}
\end{center}

The value of \texttt{Size} multiplied by the unit gives the number of
bits for the segment. Each segment can consist of zero or more bits
but the total number of bits must be a multiple of 8, or a
\texttt{badarg} run-time error will occur. Also, a segment of type
binary must have a size evenly divisible by 8.

Binaries cannot be nested.

\begin{erlang}
<<1, 17, 42>>       % <<1, 17, 42>>
<<"abc">>           % <<97, 98, 99>> (The same as <<$a, $b, $c>>)
<<1, 17, 42:16>>    % <<1,17,0,42>>
<<>>                % <<>>
<<15:8/unit:10>>    % <<0,0,0,0,0,0,0,0,0,15>>
<<(-1)/unsigned>>   % <<255>>
\end{erlang}


\section{Escape sequences}
\label{datatypes:escapeseq}
Escape sequences are allowed in strings and quoted atoms.

\begin{center}
\begin{tabular}{|>{\raggedright}p{91pt}|>{\raggedright}p{229pt}|}
\hline
\multicolumn{2}{|p{321pt}|}{Escape sequences}\tabularnewline
\hline
\textbackslash{}b & Backspace\tabularnewline
\hline
\textbackslash{}d & Delete\tabularnewline
\hline
\textbackslash{}e & Escape\tabularnewline
\hline
\textbackslash{}f & Form feed\tabularnewline
\hline
\textbackslash{}n & New line\tabularnewline
\hline
\textbackslash{}r & Carriage return\tabularnewline
\hline
\textbackslash{}s & Space\tabularnewline
\hline
\textbackslash{}t & Tab\tabularnewline
\hline
\textbackslash{}v & Vertical tab\tabularnewline
\hline
\textbackslash{}XYZ, \textbackslash{}XY, \textbackslash{}X & Character with octal
representation XYZ, XY or X\tabularnewline
\hline
\textbackslash{}\textasciicircum{}A .. \textbackslash{}\textasciicircum{}Z & Control
A to control Z\tabularnewline
\hline
\textbackslash{}\textasciicircum{}a .. \textbackslash{}\textasciicircum{}z & Control
A to control Z\tabularnewline
\hline
\textbackslash{}' & Single quote\tabularnewline
\hline
\textbackslash{}\textbf{\texttt{"}} & Double quote\tabularnewline
\hline
\textbackslash{}\textbackslash{} & Backslash\tabularnewline
\hline
\end{tabular}
\end{center}

\section{Type conversions}
There are a number of built-in functions for type conversion:

\begin{center}
\begin{tabular}{|>{\raggedright}p{63pt}|>{\raggedright}p{21pt}|>{\raggedright}p{25pt}|>{\raggedright}p{21pt}|>{\raggedright}p{21pt}|>{\raggedright}p{21pt}|>{\raggedright}p{21pt}|>{\raggedright}p{21pt}|>{\raggedright}p{24pt}|}
\hline
\multicolumn{9}{|p{243pt}|}{Type conversions}\tabularnewline
\hline
 & atom & integer & float & pid & fun & tuple & list & binary\tabularnewline
\hline
atom &  & - & - & - & - & - & X & X\tabularnewline
\hline
integer & - &  & X & - & - & - & X & X\tabularnewline
\hline
float & - & X &  & - & - & - & X & X\tabularnewline
\hline
pid & - & - & - &  & - & - & X & X\tabularnewline
\hline
fun & - & - & - & - &  & - & X & X\tabularnewline
\hline
tuple & - & - & - & - & - &  & X & X\tabularnewline
\hline
list & X & X & X & X & X & X &  & X\tabularnewline
\hline
binary & X & X & X & X & X & X & X & \tabularnewline
\hline
\end{tabular}
\end{center}

The BIF \texttt{float/1} converts integers to floats. The BIFs
\texttt{round/1} and \texttt{trunc/1} convert floats to integers.

The BIFs \texttt{Type\_to\_list/1} and \texttt{list\_to\_Type/1}
convert to and from lists.

The BIFs \texttt{term\_to\_binary/1} and \texttt{binary\_to\_term/1}
convert to and from binaries.

\begin{erlang}
atom_to_list(hello)        % "hello"
list_to_atom("hello")      % hello
float_to_list(7.0)         % "7.00000000000000000000e+00"
list_to_float("7.000e+00") % 7.00000
integer_to_list(77)        % "77"
list_to_integer("77")      % 77
tuple_to_list({a, b ,c})   % [a,b,c]
list_to_tuple([a, b, c])   % {a,b,c}
pid_to_list(self())        % "<0.25.0>"
term_to_binary(<<17>>)     % <<131,109,0,0,0,1,17>>
term_to_binary({a, b ,c})  % <<131,104,3,100,0,1,97,100,0,1,98,100,0,1,99>>
binary_to_term(<<131,104,3,100,0,1,97,100,0,1,98,100,0,1,99>>)  % {a,b,c}
term_to_binary(math:pi())  % <<131,99,51,46,49,52,49,53,57,50,54,53,51,...>>
\end{erlang}

