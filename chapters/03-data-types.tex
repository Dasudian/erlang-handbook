\chapter{Data types (terms)}
\label{datatypes}

\section{Unary data types}

\subsection{Atoms}
\label{datatypes:atom}
An \textbf{atom} is a symbolic name, also known as a
\textit{literal}. They begin with a lower-case letter or must be
enclosed within single quotes (') if it contains other characters than
alphanumeric characters, underscore (\texttt{\_}) or commercial at
(\texttt{@}).

\texttt{hello}

\texttt{phone\_number}

\texttt{'Monday'}

\texttt{'phone number'}

\texttt{'Anything inside quotes \textbackslash n\textbackslash 012'}
(see section \ref{datatypes:escapeseq})


\subsection{Booleans}
\label{datatypes:boolean}
There is no \textbf{boolean} data type in Erlang. The atoms true and
false are used instead.

\texttt{2 =< 3} $\Rightarrow$ \texttt{true} \\
\texttt{true or false} $\Rightarrow$ \texttt{true}


\subsection{Integers}
\label{datatypes:integer}
In addition to the normal way of writing \textbf{integers} Erlang
provides further notations. \$Char is the Latin-1 numeric value of the
character Char and Base\#Value is an integer of the base Base, which
must be an integer in the range 2..36. Also escape sequences are
allowed.


\texttt{42} $\Rightarrow$ \texttt{42} \\
\$A  $\Rightarrow$ \texttt{65} \\
\texttt{\$\textbackslash n} $\Rightarrow$  \texttt{10}
(see section \ref{datatypes:escapeseq}) \\
\texttt{2\#101} $\Rightarrow$ \texttt{5} \\
\texttt{16\#1f} $\Rightarrow$ \texttt{31}


\subsection{Floats}
\label{datatypes:float}
A \textbf{float} is a real number written like \texttt{Num[eExp]}
where \texttt{Num} is a decimal number between 0.01 and 10000 and
\texttt{Exp} is a signed integer.

\texttt{2.3e-3} $\Rightarrow$ \texttt{2.30000e-3}
(corresponding to 2.3*10\textsuperscript{-3})


\subsection{References}
\label{datatypes:reference}
A \textbf{reference} is a term which is unique in an Erlang runtime
system, created by the \textit{BIF} (see section \ref{functions:bifs})
\texttt{make\_ref/0}


\subsection{Ports}
\label{datatypes:port}
A \textbf{port identifier} identifies a port (see chapter \ref{ports}).


\subsection{Pids}
\label{datatypes:pid}
A \textbf{process identifier}, \textit{pid}, identifies a process (see
chapter \ref{processes}).


\subsection{Funs}
\label{datatypes:fun}
A fun identifies a \textbf{functional object} (see section
\ref{functions:funs}).


\section{Compound data types}

\subsection{Tuples}
\label{datatypes:tuple}
A \textbf{tuple} is a compound data type that holds a \textbf{fixed
number of terms} enclosed within curly braces.

\texttt{\{Term1,...,TermN\}}

Each \texttt{TermX} in the tuple is called an \textbf{element}. The
number of elements is called the \textbf{size} of the tuple.

\begin{center}
\begin{tabular}{|>{\raggedright}p{134pt}|>{\raggedright}p{186pt}|}
\hline
\multicolumn{2}{|p{321pt}|}{BIFs to manipulate tuples}\tabularnewline
\hline
size(Tuple) & Returns the size of Tuple\tabularnewline
\hline
element(N,Tuple) & Returns the N\textsuperscript{th} element in Tuple\tabularnewline
\hline
setelement(N,Tuple,Expr) & Returns a new tuple copied from Tuple except that the
N\textsuperscript{th} element is replaced by Expr\tabularnewline
\hline
\end{tabular}
\end{center}

\texttt{P = \{adam, 24, \{july, 29\}\}} $\Rightarrow$ \texttt{P} is bound to \texttt{\{adam, 24, \{july, 29\}\}} \\
\texttt{element(1, P)} $\Rightarrow$ \texttt{adam} \\
\texttt{element(3, P)} $\Rightarrow$ \texttt{ \{july,29\}} \\
\texttt{P2 = setelement(2, P, 25)} $\Rightarrow$ \texttt{\{adam, 25, \{july, 29\}\}} \\
\texttt{size(P)} $\Rightarrow$ \texttt{3} \\
\texttt{size(\{\})} $\Rightarrow$ \texttt{0} \\


\subsection{Records}
\label{datatypes:record}
A \textbf{record} is a \textit{named tuple} with named elements,
called \textbf{fields}. A record named \texttt{Rec} is defined as a
module attribute.

\begin{erlang}
-record(Rec, {Field1 [= Value1],
              ...
              FieldN [= ValueN]}).
\end{erlang}

\texttt{Rec} and \texttt{Fields} are atoms and each \texttt{FieldX}
can be given an optional default \texttt{ValueX}. This definition may
be placed among the functions of a module but before it is used. If a record is used by several modules it is advisable to put it in a
separate file for inclusion.

A new record of type \texttt{Rec} is created using an expression like this:

\begin{erlang}
#Rec{Field1=Expr1, ..., FieldK=ExprK [, _=ExprL]}
\end{erlang}

The fields need not be in the same order as in the record
definition. Fields omitted will get their respective default
values. If the final clause is used, omitted fields will get the value
\texttt{ExprL}. Fields without default values and that are omitted
will have an undefined value.

The value of a field is retrieved using the expression
\texttt{Variable\#Rec.Field}

\begin{erlang}
-module(employee).
-export([new/2]).
-record(person, {name, age, employed=erixon}).

new(Name, Age) -> #person{name=Name, age=Age}.
\end{erlang}

The function \texttt{employee:new/2} can be used in another module
which must also include the same record definition of person.

\texttt{\{P = employee:new(ernie,44)\}} $\Rightarrow$ \texttt{\{person, ernie, 44,
erixon\}} \\
\texttt{P\#person.age} $\Rightarrow$ \texttt{44} \\
\texttt{P\#person.employed} $\Rightarrow$ \texttt{erixon}

When working with records in the Erlang shell, the functions \texttt{rd(RecordName, RecordDefinition)} and \texttt{rr(Module)} could be used to define and load record definitions. Refer to the Erlang Reference Manual for more information.


\subsection{Lists}
\label{datatypes:list}
A \textbf{list} is a compound data type that holds a \textit{variable}
number of \textbf{terms} enclosed within square brackets.

\texttt{[Term1,...,TermN]}

Each term \texttt{TermX} in the list is called an
\textbf{element}. The number of elements is called the \textbf{length}
of the list. The first element is called the \textbf{head} of the list
and the remainder from the 2\textsuperscript{nd} element on is called
the \textbf{tail} of the list.

\begin{center}
\begin{tabular}{|>{\raggedright}p{90pt}|>{\raggedright}p{230pt}|}
\hline
\multicolumn{2}{|p{321pt}|}{BIFs to manipulate lists}\tabularnewline
\hline
length(List) & Returns the length of List\tabularnewline
\hline
hd(List) & Returns the 1\textsuperscript{st} element of List\tabularnewline
\hline
tl(List) & Returns List with the 1\textsuperscript{st} element removed\tabularnewline
\hline
\end{tabular}
\end{center}

The vertical bar operator (\textbar{}) separates the remainder of a
list.

\texttt{[H | T]  = [1, 2, 3, 4, 5]} $\Rightarrow$ \texttt{H=1} and \texttt{T=[2, 3, 4, 5]} \\
\texttt{[X, Y | Z] = [a, b, c, d, e]} $\Rightarrow$ \texttt{X=a}, \texttt{Y=b} and \texttt{Z=[c, d, e]}

Implicitly a list will end by an empty list, i.e. \texttt{[a, b]} is
the same as \texttt{[a, b | []]}. A list looking like \texttt{[a, b |
c]} is \textbf{badly formed} and should be avoided. Lists lend
themselves naturally to recursive programming, for example the
following functions where the first computes the sum of a list and the
second returns a list where each element has been multiplied by 2.

\begin{erlang}
sum([]) -> 0;
sum([H | T]) -> H + sum(T).

double([]) -> [];
double([H | T]) -> [H*2 | double(T)].
\end{erlang}

The operator \texttt{++} appends the second argument to the first and
returns the resulting list. The operator \texttt{--} produces a list
which is a copy of the first argument except that for each element in
the second argument, the first occurrence of this element (if any) is
removed.

\texttt{[1,2,3] ++ [4,5]} $\Rightarrow$ \texttt{[1,2,3,4,5]}
\texttt{[1,2,3,2,1,2] -- [2,1,2]} $\Rightarrow$ \texttt{[3,1,2]}

A collection of list processing functions can be found in the
\texttt{STDLIB} module lists.


\subsection{Strings}
\label{datatypes:string}
\textbf{Strings} are character strings enclosed within double quotes
but are, in fact, stored as lists.

\texttt{"abcdefghi"} is the same as \texttt{[97,98,99,100,101,102,103,104,105]}

\texttt{""} is the same as \texttt{[]}

Two adjacent strings will be concatenated into one at compile-time and
do not incur any runtime overhead.

\texttt{"string"} \texttt{"42"} $\Rightarrow$ \texttt{"string42"}


\subsection{Binaries}
\label{datatypes:binary}
A binary is a chunk of untyped memory by default a sequence of 8-bit
bytes.

\texttt{<<Elem1,...,ElemN>>}

Each \texttt{ElemX} is specified as
\texttt{Value[:Size][/TypeSpecifierList]}.

\begin{center}
\begin{tabular}{|>{\raggedright}p{45pt}|>{\raggedright}p{28pt}|>{\raggedright}p{81pt}|>{\raggedright}p{147pt}|}
\hline
\multicolumn{4}{|p{297pt}|}{{\large{}Element specification}}\tabularnewline
\hline
\multicolumn{2}{|p{75pt}|}{Value} & Size & TypeSpecifierList\tabularnewline
\hline
\multicolumn{2}{|p{75pt}|}{Should evaluate to an integer, float or binary} & Should
evaluate to an integer & A sequence of optional type specifiers, in any order,
separated by hyphens (-)\tabularnewline
\hline
\multicolumn{4}{|p{297pt}|}{T{\large{}ype specifiers}}\tabularnewline
\hline
Type & \multicolumn{2}{p{120pt}|}{integer \textbar{} float \textbar{} binary} & Default
is integer\tabularnewline
\hline
Signedness & \multicolumn{2}{p{120pt}|}{signed \textbar{} unsigned} & Default is
signed\tabularnewline
\hline
Endianness & \multicolumn{2}{p{120pt}|}{big \textbar{} little \textbar{} native} & CPU
dependent. Default is big\tabularnewline
\hline
Unit & \multicolumn{2}{p{120pt}|}{unit:IntegerLiteral} & Allowed range is 1..256.
Default is 1 for integer and float and 8 for binary\tabularnewline
\hline
\end{tabular}
\end{center}

The value of \texttt{Size} multiplied by the unit gives the number of
bits for the segment. Each segment can consist of zero or more bits
but the total number of bits must be a multiple of \texttt{8}, or a
\texttt{badarg} run-time error will occur. Also, a segment of type
binary must have a size evenly divisible by \texttt{8}.

Binaries cannot be nested.

\begin{erlang}
<<1, 17, 42>>       % <<1, 17, 42>>
<<"abc">>           % <<97, 98, 99>> (The same as <<$a, $b, $c>>)
<<1, 17, 42:16>>    % <<1,17,0,42>>
<<>>                % <<>>
<<15:8/unit:10>>    % <<0,0,0,0,0,0,0,0,0,15>>
<<(-1)/unsigned>>   % <<255>>
\end{erlang}


\section{Escape sequences}
\label{datatypes:escapeseq}
Escape sequences are allowed in strings and quoted atoms.

\begin{center}
\begin{tabular}{|>{\raggedright}p{91pt}|>{\raggedright}p{229pt}|}
\hline
\multicolumn{2}{|p{321pt}|}{E{\large{}scape sequences}}\tabularnewline
\hline
\textbackslash{}b & Backspace\tabularnewline
\hline
\textbackslash{}d & Delete\tabularnewline
\hline
\textbackslash{}e & Escape\tabularnewline
\hline
\textbackslash{}f & Form feed\tabularnewline
\hline
\textbackslash{}n & New line\tabularnewline
\hline
\textbackslash{}r & carriage Return\tabularnewline
\hline
\textbackslash{}s & Space\tabularnewline
\hline
\textbackslash{}t & Tab\tabularnewline
\hline
\textbackslash{}v & Vertical tab\tabularnewline
\hline
\textbackslash{}XYZ, \textbackslash{}XY, \textbackslash{}X & Character with octal
representation XYZ, XY or X\tabularnewline
\hline
\textbackslash{}\textasciicircum{}A .. \textbackslash{}\textasciicircum{}Z & Control
A to control Z\tabularnewline
\hline
\textbackslash{}\textasciicircum{}a .. \textbackslash{}\textasciicircum{}z & Control
A to control Z\tabularnewline
\hline
\textbackslash{}' & Single quote\tabularnewline
\hline
\textbackslash{}\textbf{\texttt{"}} & Double quote\tabularnewline
\hline
\textbackslash{}\textbackslash{} & Backslash\tabularnewline
\hline
\end{tabular}
\end{center}

\section{Type conversions}
There are a number of BIFs for type conversions.

\begin{center}
\begin{tabular}{|>{\raggedright}p{63pt}|>{\raggedright}p{21pt}|>{\raggedright}p{21pt}|>{\raggedright}p{21pt}|>{\raggedright}p{21pt}|>{\raggedright}p{21pt}|>{\raggedright}p{21pt}|>{\raggedright}p{21pt}|>{\raggedright}p{22pt}|}
\hline
\multicolumn{9}{|p{237pt}|}{T{\large{}ype conversions}}\tabularnewline
\hline
 & atom & integer & float & pid & fun & tuple & list & binary\tabularnewline
\hline
atom &  & - & - & - & - & - & X & X\tabularnewline
\hline
integer & - &  & X & - & - & - & X & X\tabularnewline
\hline
float & - & X &  & - & - & - & X & X\tabularnewline
\hline
pid & - & - & - &  & - & - & X & X\tabularnewline
\hline
fun & - & - & - & - &  & - & X & X\tabularnewline
\hline
tuple & - & - & - & - & - &  & X & X\tabularnewline
\hline
list & X & X & X & X & X & X &  & X\tabularnewline
\hline
binary & X & X & X & X & X & X & X & \tabularnewline
\hline
\end{tabular}
\end{center}

The BIF \texttt{float/1} converts integers to floats. The BIFs
\texttt{round/1} and \texttt{trunc/1} convert floats to integers.

The BIFs \texttt{Type\_to\_list/1} and \texttt{list\_to\_Type/1}
convert to and from lists.

The BIFs \texttt{term\_to\_binary/1} and \texttt{binary\_to\_term/1}
convert to and from binaries.

\begin{erlang}
atom_to_list(hello)        % "hello"
list_to_atom("hello")      % hello
float_to_list(7.0)         % "7.00000000000000000000e+00"
list_to_float("7.000e+00") % 7.00000
integer_to_list(77)        % "77"
list_to_integer("77")      % 77
tuple_to_list({a, b ,c})   % [a,b,c]
list_to_tuple([a, b, c])   % {a,b,c}
pid_to_list(self())        % "<0.25.0>"
term_to_binary(<<17>>)     % <<131,109,0,0,0,1,17>>
term_to_binary({a, b ,c})  % <<131,104,3,100,0,1,97,100,0,1,98,100,0,1,99>>
binary_to_term(<<131,104,3,100,0,1,97,100,0,1,98,100,0,1,99>>)  % {a,b,c}
term_to_binary(math:pi())  % <<131,99,51,46,49,52,49,53,57,50,54,53,51,...>>
\end{erlang}

