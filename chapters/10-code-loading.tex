\chapter{Code loading}
\label{code}

Erlang supports code updating in a running system. Code replacement is
performed at module level.

The code of a module can exist in two versions in a system:
\textbf{current} and \textbf{old}. When a module is
loaded into the system for the first time, the code becomes
\textit{current}. If a new instance of the module is loaded, the code
of the previous instance becomes \textit{old} and the new instance
becomes \textit{current}. Normally a module is automatically loaded
the first time a function in it is called. If the module is already
loaded then it must explicitly be loaded again to a new version.

Both old and current code are valid, and may be used
concurrently. Fully qualified function calls will always refer to the
current code. However, the old code may still be run by other
processes.

If a third instance of the module is loaded, the code server will
remove (\textit{purge}) the old code and any processes lingering in it
are terminated. Then the third instance becomes \textit{current} and
the previously current code becomes \textit{old}.

To change from old code to current code, a process must make a fully
qualified function call.

\begin{erlang}
-module(mod).
-export([loop/0]).

loop() ->
    receive
        code_switch ->
            mod:loop();
        Msg ->
            ...
            loop()
    end.
\end{erlang}

To make the process change code, send the message
\texttt{code\_switch} to it. The process then will make a fully
qualified call to \texttt{mod:loop()} and change to the current
code. Note that \texttt{mod:loop/0} must be exported.