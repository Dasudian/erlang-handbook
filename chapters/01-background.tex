\chapter{Background, or Why Erlang is that it is}
\label{background}

Erlang is the result of a project at Ericsson's Computer Science Laboratory to
improve the programming of telecommunication applications.  A critical
requirement was supporting the characteristics of such applications, that include:

\begin{itemize}
\item Massive concurrency

\item Fault-tolerance

\item Isolation

\item Dynamic code upgrading at runtime

\item Transactions
\end{itemize}

Throughout the whole of Erlang's history the development process has
been extremely pragmatic. The characteristics and properties of the
types of systems in which Ericsson was interested drove Erlang's
development.  These properties were considered to be so
fundamental that it was decided to build support for them into the
language itself, rather than in libraries.  Because of the pragmatic development
process, rather than a result of prior planning, Erlang ``became'' a functional language --- since
the features of functional languages fitted well with the properties of the systems being developed.

