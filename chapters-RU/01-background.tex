\chapter[Вступление, или почему Erlang такой, как он есть]{Вступление}
\section*{или почему Erlang такой, как он есть}
\label{background}

Erlang --- это результат проекта по улучшению программирования приложений для 
телекоммуникаций в Лаборатории Компьютерных Наук (CSLab или Computer Science Lab)
компании Ericsson. Критически важным требованием была поддержка характеристик 
этих приложений, таких, как: 

\begin{itemize}
	\item Массивная параллельность
	\item Устойчивость к сбоям
	\item Изоляция
	\item Динамическое обновление кода во время его исполнения
	\item Транзакционность
\end{itemize}

В течение всей истории Erlang процесс его разработки был исключительно 
прагматичным. Характеристики и свойства видов систем, в которых была заинтересована
компания Ericsson, прямым образом влияли на ход разработки Erlang. Эти свойства
считались настолько фундаментальными, что было решено поддержку для них встроить
прямо в язык, вместо дополнительных библиотек. По причине прагматичного процесса
разработки вместо предварительного планирования, Erlang "<стал"> функциональным
языком --- поскольку свойства функциональных языков очень хорошо подходили к
свойствам систем, которые разрабатывались.

