\chapter{Функции}

\section{Определение функции}

Функция определяется, как последовательность из одного или нескольких  
\textbf{уравнений функции}. Имя функции должно быть атомом.

\vspace*{4pt}
\begin{erlangru}
Функция(Образец11,...,Образец1N)
  [when ОхранныеВыражения1] -> ТелоФункции1;
...;
...;
Функция(ОбразецK1,...,ОбразецKN)
  [when ОхранныеВыраженияK] -> ТелоФункцииK.
\end{erlangru}
\vspace*{4pt}

Уравнения функции разделены точками с запятой (\texttt{;}) и последнее уравнение
завершается точкой (\texttt{.}). Уравнение функции состоит из \textbf{заголовка
уравнения} и \textbf{тела уравнения функции}, разделённых стрелкой 
(\texttt{->}). 

Заголовок уравнения состоит из имени функции (атома), списка аргументов, 
заключённого в скобки, и необязательного списка охранных выражений, начинающихся
с ключевого слова \texttt{when}.  Каждый аргумент функции --- образец.  Тело 
функции состоит из последовательности выражений, разделённых запятыми (\texttt{,}).

\vspace*{4pt}
\begin{erlangru}
Выражение1,
...,
ВыражениеM
\end{erlangru}
\vspace*{4pt}

Количество аргументов \texttt{N} ещё называется \textbf{арностью} функции.
Функция уникально определяется именем модуля, именем функции и своей арностью. 
Две разные функции в одном модуле, имеющие разную арность, могут иметь одно и то
же имя.  \texttt{Функция} в \texttt{Модуле} с арностью \texttt{N} часто может
записываться так: \texttt{Модуль:Функция/N}.

\vspace*{4pt}
\begin{erlang}
-module(mathStuff).
-export([area/1]).

area({square, Side}) -> Side * Side;
area({circle, Radius}) -> math:pi() * Radius * Radius;
area({triangle, A, B, C}) ->
    S = (A + B + C)/2,
    math:sqrt(S*(S-A)*(S-B)*(S-C)).
\end{erlang}


% push this section onto the next page (orphan line)
\newpage


\section{Вызовы функций}

Функция вызывается с помощью записи:

\begin{erlangru}
[Модуль:]Функция(Выражение1, ..., ВыражениеN)
\end{erlangru}

Выражение \texttt{Модуль} должно вычисляться в имя модуля (или быть атомом) и
выражение \texttt{Функция} должно вычисляться в имя функции или в 
\emph{анонимную функцию}. При вызове функции в другом модуле, следует указать 
имя модуля и функция должна быть экспортирована. Такой вызов будет называться
\textbf{полностью определённым вызовом функции}.

\begin{erlang}
lists:keysearch(Name, 1, List)
\end{erlang}

Имя модуля может быть опущено, если \texttt{Функция} вычисляется в имя 
локальной функции, импортированной функции или авто-импортированной встроенной
(BIF)  функции. Такой вызов называется \textbf{неявно определённым вызовом
функции}.

Перед тем, как вызвать функцию, вычисляются аргументы \texttt{ExprX}.  Если 
функция не может быть найдена, то возникает ошибка времени исполнения
\texttt{undef}. Если уравнений функции несколько, они последовательно
сканируются до тех пор, пока не будет найдено подходящее уравнение, такое, что
образцы в заголовке уравнения успешно могут быть сопоставлены с данными
аргументами и что охранное выражение  (если оно задано) равно \texttt{true}.
Если такое уравнение не может быть найдено, возникает ошибка времени исполнения 
\texttt{function\_clause}.

Если совпадающее уравнение найдено, то вычисляется соответствующее тело 
\linebreak
функции, то есть выражения в теле функции вычисляются одно за другим и 
возвращается результат последнего выражения.

Полностью определённое имя функции должно быть использовано, если вызывается 
встроенная функция с таким же именем (см. секцию \ref{functions:bifs} о 
встроенных BIF функциях).  Компилятор не разрешит определить функцию с таким же
именем, как другая импортированная функция. При вызове локальной функции есть
разница между использованием неявно или полностью определённого имени функции,
поскольку второе всегда относится к последней версии модуля (см. главу
\ref{code} о модулях и версиях).


\section{Выражения}
\label{functions:expressions}

\textbf{Выражение} это терм либо вызов оператора, результатом которого будет 
терм, например:

\begin{erlangru}
Терм
операция Выражение
Выражение1 операция Выражение2
(Выражение)
begin
   Выражение1,
   ...,
   ВыражениеM            % нет запятой (,) перед end
end
\end{erlangru}

В наличии имеются как \emph{унарные} так и \emph{бинарные} операторы. Простейшая
форма выражения --- это терм, например \emph{целое число}, \emph{число с 
плавающей точкой}, \emph{атом}, \emph{строка}, \emph{список} или \emph{кортеж},
и возвращаемое оператором значение тоже терм. Выражение может содержать
\emph{макрос} или операции над \emph{записью}, которые будут развёрнуты во
время компиляции.

Выражения в скобках полезны для изменения порядка вычисления операторов (см.  
раздел \ref{functions:expressions:precedence}):

\begin{erlang}
1 + 2 * 3           % 7
(1 + 2) * 3         % 9
\end{erlang}

Блочные выражения, заключённые в операторные скобки между \texttt{begin...end},
могут использоваться для группировки последовательности выражений и возвращают 
значение, равное значению последнего выражения внутри \texttt{ВыражениеM}.

Все вложенные подвыражения вычисляются до главного выражения, но порядок, в 
котором происходит вычисление вложенных, не определён стандартом.

Многие операторы могут применяться только к аргументам определённого типа. 
Например, арифметические операторы могут только применяться к целым числам или 
числам с плавающей точкой. Аргумент неверного типа вызовет ошибку времени 
выполнения \texttt{badarg}.


\subsection{Сравнение термов}

\begin{verbatim}
Выражение1 оператор Выражение2
\end{verbatim}

\textbf{Сравнение термов} возвращает \emph{булево} (логическое) значение, в 
форме атомов \texttt{true} или \texttt{false}.

\begin{center}
\begin{tabular}{|>{\raggedright}p{40pt}|>{\raggedright}p{185pt}|>{\raggedright}p{26pt}|>{\raggedright}p{135pt}|}
\hline
\multicolumn{4}{|p{297pt}|}{Операторы сравнения}\tabularnewline
\hline
\texttt{==} & Равно (с приведением типа) &
\texttt{=<} & Меньше или равно \tabularnewline
\hline
\texttt{/=} & Не равно (с приведением типа) &
\texttt{<} & Меньше, чем \tabularnewline
\hline
\texttt{=:=} & Точно равно (с учётом типа) &
\texttt{>}= & Больше или равно \tabularnewline
\hline
\texttt{=/=} & Точно не равно (с учётом типа) &
\texttt{>} & Больше, чем \tabularnewline
\hline
\end{tabular}
\end{center}

\begin{erlang}
1==1.0              % true
1=:=1.0             % false
1 > a               % false
\end{erlang}

Аргументы оператора сравнения могут иметь разные типы данных. В таком случае 
действует следующий порядок сравнения:

\texttt{число < атом < ссылка < анонимная функция < порт < pid < кортеж < 
	список < двоичные данные}

Списки сравниваются поэлементно. Кортежи сравниваются по размеру, два кортежа 
одного размера сравниваются поэлементно. При сравнении целого числа и числа с 
плавающей точкой, целое сначала приводится к числу с плавающей точкой. В случае 
использования точного равенства \texttt{=:=} или \texttt{=/=} тип числа не 
изменяется и учитывается в равенстве.


\subsection{Арифметические выражения}

\begin{erlangru}
оператор Выражение
Выражение1 оператор Выражение2
\end{erlangru}

\textbf{Арифметическое выражение} возвращает результат после применения 
оператора.

\begin{center}
\begin{tabular}{|>{\raggedright}p{35pt}|>{\raggedright}p{185pt}|>{\raggedright}p{128pt}|}
\hline
\multicolumn{3}{|p{309pt}|}{Арифметические операторы}\tabularnewline
\hline
\texttt{+}  & Унарный плюс  &
\texttt{Integer} \textbar{} \texttt{Float} \tabularnewline
\hline
\texttt{-}  & Унарный минус  &
\texttt{Integer} \textbar{} \texttt{Float} \tabularnewline
\hline
\texttt{+}  & Сложение &
\texttt{Integer} \textbar{} \texttt{Float}\tabularnewline
\hline
\texttt{-}  & Вычитание &
\texttt{Integer} \textbar{} \texttt{Float}\tabularnewline
\hline
\texttt{*}  & Умножение &
\texttt{Integer} \textbar{} \texttt{Float}\tabularnewline
\hline
\texttt{/}  & Деление с плавающей точкой &
\texttt{Integer} \textbar{} \texttt{Float}\tabularnewline
\hline
\texttt{bnot}  & Унарное битовое НЕ &
\texttt{Integer} \tabularnewline
\hline
\texttt{div}  & Целочисленное деление &
\texttt{Integer}\tabularnewline
\hline
\texttt{rem}  & Целочисленный остаток от деления X на Y  &
\texttt{Integer} \tabularnewline
\hline
\texttt{band}  & Битовое И & \texttt{Integer}\tabularnewline
\hline
\texttt{bor}  & Битовое ИЛИ & \texttt{Integer} \tabularnewline
\hline
\texttt{bxor}  & Исключающее битовое ИЛИ &
\texttt{Integer}\tabularnewline
\hline
\texttt{bsl}  & Арифметический (с учётом знака) битовый сдвиг влево &
\texttt{Integer} \tabularnewline
\hline
\texttt{bsr}  & Битовый сдвиг вправо &
\texttt{Integer}\tabularnewline
\hline
\end{tabular}
\end{center}

\begin{erlang}
+1                  % 1
4/2                 % 2.00000
5 div 2             % 2
5 rem 2             % 1
2#10 band 2#01      % 0
2#10 bor 2#01       % 3
\end{erlang}


\subsection{Логические (булевы) выражения}

\begin{erlangru}
оператор Выражение
Выражение1 оператор Выражение2
\end{erlangru}

\textbf{Логическое выражение} возвращает значение \texttt{true} или 
\texttt{false} после применения оператора.

\begin{center}
\begin{tabular}{|>{\raggedright}p{79pt}|>{\raggedright}p{241pt}|}
\hline
\multicolumn{2}{|p{321pt}|}{Логические (булевы) операторы}\tabularnewline
\hline
\texttt{not}  & Унарное логическое НЕ (отрицание) \tabularnewline
\hline
\texttt{and}  & Логическое И \tabularnewline
\hline
\texttt{or}  & Логическое ИЛИ \tabularnewline
\hline
\texttt{xor}  & Логическое исключающее ИЛИ \tabularnewline
\hline
\end{tabular}
\end{center}

\begin{erlang}
not true            % false
true and false      % false
true xor false      % true
\end{erlang}


\subsection{Умные логические выражения}

\begin{erlangru}
Выражение1 orelse Выражение2
Выражение1 andalso Выражение2
\end{erlangru}

В этих логические выражениях второй операнд вычисляется только в том случае, 
если его значение необходимо для конечного результата. В случае с
\texttt{orelse} \texttt{Выражение2} будет вычислено, только если
\texttt{Выражение1} равняется \texttt{false}. В случае с \texttt{andalso}
\texttt{Выражение2} будет вычислено, только если \texttt{Выражение1} равняется 
\texttt{true}.

\begin{erlang}
if A >= 0 andalso math:sqrt(A) > B -> ...

if is_list(L) andalso length(L) == 1 -> ...
\end{erlang}


\subsection{Приоритет операторов}
\label{functions:expressions:precedence}

В выражении, состоящем из подвыражений, операторы будут применяться согласно 
определённому порядку, который называется \textbf{приоритетом операторов}:

\begin{center}
\begin{tabular}{|>{\raggedright}p{221pt}|>{\raggedright}p{125pt}|}
\hline
\multicolumn{2}{|p{321pt}|}{Приоритет операторов (от высшего к низшему)} \tabularnewline
\hline
\texttt{:} ~ &  \tabularnewline
\hline
\texttt{\#} ~ &  \tabularnewline
\hline
\texttt{Unary + - bnot not ~} &  \tabularnewline
\hline
\texttt{/ * div rem band and}  & Левоассоциативный \tabularnewline
\hline
\texttt{+ - bor bxor bsl bsr or xor} & Левоассоциативный \tabularnewline
\hline
\texttt{++ --}  & Правоассоциативный \tabularnewline
\hline
\texttt{== /= =< < >= > =:= =/=} & \tabularnewline
\hline
\texttt{andalso}  &  \tabularnewline
\hline
\texttt{orelse} &  \tabularnewline
\hline
\texttt{= !}  & Правоассоциативный \tabularnewline
\hline
\texttt{catch ~} &  \tabularnewline
\hline
\end{tabular}
\end{center}

Оператор с наивысшим приоритетом вычисляется первым. Операторы с одинаковым 
приоритетом вычисляются согласно их \textbf{ассоциативности}. Левоассоциативные
операторы вычисляются слева направо, правоассоциативные --- наоборот, справа
налево:

\texttt{6 + 5 * 4 - 3 / 2  \resultingin  6 + 20 - 1.5  \resultingin  26 - 1.5  
	\resultingin  24.5}



\section{Составные выражения}


\subsection{If}

\begin{erlangru}
if
    ОхранноеВыражение1 ->
        Тело1;
    ...;
    ОхранноеВыражениеN ->
        ТелоN      % Заметьте, нет точки с запятой (;) перед end
end
\end{erlangru}

Ветки выражения \texttt{if} сканируются последовательно сверху вниз, пока не 
встретится \texttt{ОхранноеВыражениеX}, результатом которого будет
\texttt{true}. Соотвествующее \texttt{ТелоX} (последовательность выражений,
разделённых  запятыми) будет вычислено. Возвращаемое \texttt{ТеломX} значение
будет результатом всего выражения \texttt{if}.

Если ни одно из охранных выражений не вычисляется в \texttt{true}, то возникнет
ошибка времени выполнения \texttt{if\_clause}. При необходимости можно 
использовать\linebreak
охранное выражение \texttt{true} в последней ветке \texttt{if}, 
поскольку такое охранное выражение всегда срабатывает, если другие не сработали 
и называется "<срабатывающим для всех значений"> (catch all).

\begin{erlang}
is_greater_than(X, Y) ->
    if
        X>Y ->
            true;
        true ->                 % работает как ветка 'else'
            false
    end
\end{erlang}

Следует заметить, что сопоставление с образцом в уравнениях функций может почти
всегда использоваться для замены \texttt{if}. Чрезмерное использование 
\texttt{if} внутри функций считается плохой практикой.


\subsection{Case}

Выражения \texttt{case} используются для сопоставления с образцом прямо в коде, 
подобно тому, как сопоставляются аргументы в заголовках функций.

\begin{erlangru}
case Выражение of
    Выражение1 [when ОхранноеВыражение1] ->
        Тело1;
        ...;
    ВыражениеN [when ОхранноеВыражениеN] ->
        ТелоN       % Заметьте, нет точки с запятой (;) перед end
end
\end{erlangru}

\texttt{Выражение} вычисляется и \texttt{Образец1}...\texttt{ОбразецN} 
последовательно сопоставляются с результатом. Если сопоставление проходит 
успешно, и необязательное \texttt{Охран\-ное\-ВыражениеХ} равно \texttt{true}, 
тогда вычисляется соответствующее \texttt{ТелоХ}. Возвращаемое значение
\texttt{ТелаХ} является  результатом всего выражения \texttt{case}.

В случае, если ни один из образцов и их охранных выражений не подходит, 
возникает ошибка времени выполнения \texttt{case\_clause}.

\begin{erlang}
is_valid_signal(Signal) ->
    case Signal of
        {signal, _What, _From, _To} ->
            true;
        {signal, _What, _To} ->
            true;
        _Else ->                % 'catch all'
            false
    end.
\end{erlang}


\subsection{Генераторы списков}

Генераторы списков аналогичны предикатам \texttt{setof} и \texttt{findall}
в языке Prolog.

\begin{erlangru}
[Выражение || Квалификатор1,...,КвалификаторN]
\end{erlangru}

\texttt{Выражение} --- произвольное выражение, и каждый 
\texttt{КвалификаторX} --- это либо \textbf{генератор} (источник данных) 
либо \textbf{фильтр}. Генератор записывается, как:

\begin{erlangru}
Образец <- ВыражениеСписок
\end{erlangru}

где \texttt{ВыражениеСписок} должно быть выражением, результатом которого 
является список термов. Фильтр --- это выражение, результатом которого является 
\texttt{true} или \texttt{false}. Переменные внутри генератора списков 
\emph{затеняют} переменные с такими же именами, принадлежащие функции, 
окружающей генератор.

Квалификаторы вычисляются слева направо, генераторы создают значения и\linebreak
фильтры отбирают нужные из них. Генератор списков возвращает список, в котором
элементы являются результатом вычисления \texttt{Выражения} для каждой
комбинации результирующих значений.

\begin{minted}{console}
> [{X, Y} || X <- [1,2,3,4,5,6], X > 4, Y <- [a,b,c]].
[{5,a},{5,b},{5,c},{6,a},{6,b},{6,c}]
\end{minted}


\section{Охранные последовательности}

\textbf{Охранная последовательность} --- это набор \textbf{охранных выражений},
разделённых точками с запятой (\texttt{;}). Охранная последовательность 
равняется \texttt{true}, если как минимум одно из составляющих её охранных
выражений равно \texttt{true}.

\begin{erlangru}
Охрана1; ...; ОхранаK
\end{erlangru}

\textbf{Охрана} --- это множество \textbf{охранных выражений}, разделённых 
запятыми (\texttt{,}). Результатом будет \texttt{true} если все охранные
выражения равняются \texttt{true}.

\begin{erlangru}
ОхранноеВыражение1, ..., ОхранноеВыражениеN
\end{erlangru}

В \textbf{охранных выражениях}, которые иногда ещё называются охранными тестами,
разрешены не любые выражения Erlang, а ограниченный набор, выбранный авторами 
Erlang, поскольку важно, чтобы ход вычисления охранного выражения не имел 
побочных эффектов.

\begin{center}
\begin{tabular}{|>{\raggedright}p{210pt}|>{\raggedright}p{210pt}|}
\hline
\multicolumn{2}{|p{420pt}|}{Разрешённые охранные выражения:}\tabularnewline
\hline
\multicolumn{2}{|p{420pt}|}{Атом \texttt{true};}\tabularnewline
\hline
\multicolumn{2}{|p{420pt}|}{Другие константы (термы, связанные переменные), все 
	считаются равными \texttt{false};}\tabularnewline
\hline
\multicolumn{2}{|p{420pt}|}{Сравнения термов;}\tabularnewline
\hline
\multicolumn{2}{|p{420pt}|}{Арифметические и логические 
	выражения;}\tabularnewline
\hline
\multicolumn{2}{|p{420pt}|}{Вызовы встроенных (BIF) функций, перечисленные 
	ниже:}\tabularnewline
\hline
Функции проверки типа & Другие позволенные функции:\tabularnewline
\hline
\texttt{is\_atom/1} & \texttt{abs(Integer} \textbar{} \texttt{Float)}\tabularnewline
\hline
\texttt{is\_constant/1} & \texttt{float(Term)}\tabularnewline
\hline
\texttt{is\_integer/1} & \texttt{trunc(Integer} \textbar{} \texttt{Float)}\tabularnewline
\hline
\texttt{is\_float/1} & \texttt{round(Integer} \textbar{} \texttt{Float)}\tabularnewline
\hline
\texttt{is\_number/1} & \texttt{size(Tuple} \textbar{} \texttt{Binary)}\tabularnewline
\hline
\texttt{is\_reference/1} & \texttt{element(N, Tuple)}\tabularnewline
\hline
\texttt{is\_port/1} & \texttt{hd(List)}\tabularnewline
\hline
\texttt{is\_pid/1} & \texttt{tl(List)}\tabularnewline
\hline
\texttt{is\_function/1} & \texttt{length(List)}\tabularnewline
\hline
\texttt{is\_tuple/1} & \texttt{self()}\tabularnewline
\hline
\texttt{is\_record/2} Второй аргумент имя записи & 
	\texttt{node(})\tabularnewline
\hline
\texttt{is\_list/1} & \texttt{node(Pid} \textbar{} \texttt{Ref} \textbar 
	\texttt{Port)}\tabularnewline
\hline
\texttt{is\_binary/1} & \tabularnewline
\hline
\end{tabular}
\end{center}

Небольшой пример

\begin{erlang}
fact(N) when N>0 ->             % Заголовок первого уравнения
    N * fact(N-1);              % Тело первого уравнения
fact(0) ->                      % Заголовок второго уравнения
    1.                          % Тело второго уравнения
\end{erlang}


\section{Хвостовая рекурсия}

Если последнее выражение тела функции является вызовом функции, то выполняется
\textbf{хвостовой рекурсивный} вызов так, что ресурсы системы (например, стек
вызовов) не расходуются. Это означает, что можно создать бесконечный цикл, 
такой, как, например, сервер, если использовать хвостовые рекурсивные вызовы.

Функция \texttt{fact/1} выше может быть переписана для использования хвостовой
рекурсии следующим образом:

 \begin{erlang}
fact(N) when N>1 -> fact(N, N-1);
fact(N) when N==1; N==0 -> 1.

fact(F,0) -> F;       % Переменная F используется как аккумулятор
fact(F,N) -> fact(F*N, N-1).
\end{erlang}



\section{Анонимные функции}
\label{functions:funs}

Ключевое слово \textbf{fun} определяет \emph{функциональный объект}. Анонимные 
функции делают возможным передавать целую функцию, а не только её имя, в 
качестве аргумента. Выражение, определяющее анонимную функцию, начинается с
ключевого слова \texttt{fun} и заканчивается ключевым словом \texttt{end}
вместо точки (\texttt{.}).  Между ними должно находиться обычное объявление
функции, за тем исключением, что имя её не пишется.

\begin{erlangru}
fun
    (Образец11,...,Образец1N) [when Охрана1] ->
        ТелоУравнения1;
        ...;
    (ОбразецK1,...,ОбразецKN) [when ОхранаK] ->
        ТелоУравненияK
end
\end{erlangru}

Переменные в заголовке анонимной функции \emph{затеняют} переменные в уравнении
функции, которое окружает \texttt{fun}, но переменные, связанные в теле 
\texttt{fun} являются для него локальными.  Возвращаемое выражением \texttt{fun 
Имя/N} значение является функцией. Выражение \texttt{fun Имя/N} эквивалентно 
следующей записи:

\begin{erlangru}
fun (Аргумент1,...,АргументN) -> Имя(Аргумент1,...,АргументN) end
\end{erlangru}

Выражение \texttt{fun Модуль:Функция/Арность} также разрешено, но только если 
\texttt{Функция} экспортирована из \texttt{Модуля}.

\pagebreak
\begin{erlang}
Fun1 = fun (X) -> X+1 end.
Fun1(2)         % результат: 3

Fun2 = fun (X) when X>=1000 -> big; (X) -> small end.
Fun2(2000)      % результат: big
\end{erlang}

Поскольку функция, созданная  \texttt{fun} анонимна, то есть не имеет имени в 
определении функции, то для определения рекурсивной функции следует сделать два
шага.  Пример ниже показывает, как определить рекурсивную функцию 
\texttt{sum(List)} (смотрите раздел \ref{datatypes:list}) как анонимную с 
помощью \texttt{fun}. Такой подход ещё называется \emph{Y-комбинатор}.

\begin{erlang}
Sum1 = fun ([], _Foo) -> 0;([H|T], Foo) -> H + Foo(T, Foo) end.
Sum = fun (List) -> Sum1(List, Sum1) end.
Sum([1,2,3,4,5])    % 15
\end{erlang}

Определение \texttt{Sum1} сделано так, что оно принимает \emph{само себя} в
качестве аргумента, сопоставляется с \texttt{\_Foo} (пустым списком) или
\texttt{Foo}, который затем рекурсивно вызывается.  Определение \texttt{Sum} 
вызывает \texttt{Sum1}, также передавая \texttt{Sum1} в качестве аргумента.

\textbf{Примечание:} В Erlang версии R17 эта проблема устранена и анонимные 
функции могут ссылаться сами на себя.



\section{Встроенные функции (BIF)}
\label{functions:bifs}

\textbf{Встроенные функции}, или BIF (built-in functions) --- это функции, 
реализованные на языке C, на котором также написана система Erlang, и делают 
вещи, которые трудно или невозможно реализовать на языке Erlang.  Большинство
встроенных функций принадлежат модулю \texttt{erlang}, но есть некоторые,
принадлежащие и другим модулям, таким как \texttt{lists} и \texttt{ets}. 
Используемые чаще всего встроенные функции, принадлежащие модулю 
\texttt{erlang}, импортируются во все ваши модули автоматически, то есть перед
ними не требуется писать имя модуля.

\begin{center}
\begin{tabular}{|>{\raggedright}p{135pt}|>{\raggedright}p{290pt}|}
\hline
\multicolumn{2}{|p{321pt}|}{Некоторые полезные встроенные функции} \tabularnewline
\hline
\texttt{date()} &
Возвращает сегодняшнюю дату в формате \texttt{\{Год, Месяц, День\}} \tabularnewline
\hline
\texttt{now()} &
Возвращает текущее время в микросекундах. Точность и реализация зависит от 
операционной системы \tabularnewline
\hline
\texttt{time()} &
Возвращает текущее время в формате \texttt{\{Часы, Минуты, Секунды\}}. Реализация зависит от операционной системы \tabularnewline
\hline
\texttt{halt()} & 
Аварийно останавливает работу Erlang-системы \tabularnewline
\hline
\texttt{processes()} & 
Возвращает список всех Erlang-процессов в системе \tabularnewline
\hline
\texttt{process\_info(Pid)} & 
Возвращает словарь, содержащий информацию о процессе \texttt{Pid} \tabularnewline
\hline
\texttt{Модуль:module\_info()} & 
Возвращает словарь, содержащий информацию о \texttt{Модуле} \tabularnewline
\hline
\end{tabular}
\end{center}

\textbf{Словарь} --- это список кортежей в формате \texttt{\{Ключ, Значение\}}
	(см. также раздел \ref{processes:dicts}).

\begin{erlang}
size({a, b, c})             % 3
atom_to_list('Erlang')      % "Erlang"
date()                      % {2013,5,27}
time()                      % {01,27,42}
\end{erlang}
