\chapter{Порты и драйверы портов}
\label{ports}

\textbf{Порты} предоставляют байто-ориентированный интерфейс к внешним 
программам и связывается с процессами Erlang посылая и принимая сообщения в
виде списков байтов. Процесс Erlang, который создаёт порт, называется
\textbf{владельцем порта} или \textbf{подключенным к порту процессом}.  Все
коммуникации в и из порта должны пройти через владельца порта.  Если владелец
порта завершает работу, порт тоже закроется (а также и внешняя программа,
подключенная к порту, если она была правильно написана и среагирует на закрытие 
ввода/вывода).

Внешняя программа является другим процессом операционной системы. По умолчанию, 
она должна считывать данные из стандартного входа (файловый дескриптор 0) и
отвечать на стандартный вывод (файловый дескриптор 1).  Внешняя программа
должна завершать свою работу когда порт закрывается (ввод/вывод закрывается).


\section{Драйверы портов}

Драйверы портов обычно пишутся на языке С и динамически подключаются к системе
исполнения Erlang. Встроенный драйвер ведёт себя как порт и называется 
\textbf{драйвером порта}.  Однако, ошибка в драйвере порта может привести к 
нестабильности во всей системе Erlang, утечкам памяти, зависаниям и краху 
системы.


\section{Встроенные функции для портов}

\begin{center}
\begin{tabular}{|>{\raggedright}p{140pt}|>{\raggedright}p{300pt}|}
\hline
\multicolumn{2}{|p{440pt}|}{Функция для создания порта} \tabularnewline
\hline
\texttt{open\_port(ИмяПорта, НастройкиПорта)} & 
Возвращает \textbf{идентификатор порта} \texttt{Порт}, как результат открытия 
нового Erlang-порта.  Сообщения могут быть отправлены в и получены через 
идентификатор порта, так же как это можно делать с идентификаторами процессов.
Идентификаторы портов могут быть связаны с процессами, или зарегистрированы под
каким-либо именем с помощью \texttt{link/1} и \texttt{register/2}. \tabularnewline
\hline
\end{tabular}
\end{center}

\texttt{ИмяПорта} обычно является кортежем вида \texttt{\{spawn,Команда\}}, где
строка \texttt{Команда} является именем внешней программы.  Внешняя программа 
выполняется за пределами Erlang-системы, если только не найден драйвер порта с
именем \texttt{Команда}.  Если драйвер найден, он будет активирован вместо 
команды.

\texttt{НастройкиПорта} --- это список настроек (опций) для порта. Список обычно
содержит как минимум один кортеж \texttt{\{packet,N\}}, указывающий, что данные,
пересылаемые между портом и внешней программой, предваряются N-байтовым 
индикатором длины.  Разрешённые значения для \texttt{N} --- 1, 2 или 4.  Если 
двоичные данные должны использоваться вместо списков байтов, то должна быть 
включена опция \texttt{binary}.

Владелец порта Pid связывается с \texttt{Портом} с помощью отправки и получения
Erlang-сообщений.  (Любой процесс может послать сообщение в порт, но ответы от 
порта всегда будут отправлены только владельцу порта).

\begin{center}
\begin{tabular}{|>{\raggedright}p{140pt}|>{\raggedright}p{300pt}|}
\hline
\multicolumn{2}{|p{440pt}|}{Сообщения, отсылаемые в порт}\tabularnewline
\hline
\texttt{\{Pid, \{command, Данные\}\}}  & 
Посылает \texttt{Данные} в порт. \tabularnewline
\hline
\texttt{\{Pid, close\}}  & 
Закрывает порт. Если порт был открыт, он отвечает сообщением 
\texttt{\{Порт, closed\}}, когда все буферы были сброшены и порт закрылся.
\tabularnewline
\hline
\texttt{\{Pid, \{connect, НовыйPid\}\}}  & 
Устанавливает владельца \texttt{Порта} равным \texttt{НовомуPid}. Если порт был
открыт, он отвечает сообщением \texttt{\{Порт, connected\}} старому владельцу. 
Заметьте, что старый владелец порта остаётся связанным с портом, тогда как новый
--- нет. \tabularnewline
\hline
\end{tabular}
\end{center}

Данные должны быть списком ввода-вывода. \textbf{Список ввода-вывода} (iolist) 
--- это либо двоичные данные, либо смешанный (возможно вложенный) список
двоичных данных и целых чисел в диапазоне от 0 до 255.

\begin{center}
\begin{tabular}{|>{\raggedright}p{140pt}|>{\raggedright}p{300pt}|}
\hline
\multicolumn{2}{|p{440pt}|}{Сообщения, получаемые из порта}\tabularnewline
\hline
\texttt{\{Порт, \{data, Данные\}\}}  & 
Данные получены от внешней программы \tabularnewline
\hline
\texttt{\{Порт, closed\}}  & 
Ответ на команду \texttt{Порт ! \{Pid,close\}} \tabularnewline
\hline
\texttt{\{Порт, connected\}}  & 
Ответ на команду \texttt{Порт ! \{Pid,\{connect, NewPid\}\}} \tabularnewline
\hline
\texttt{\{'EXIT', Порт, Причина\}}  &
Присылается, если порт был отключен по какой-либо причине. \tabularnewline
\hline
\end{tabular}
\end{center}

Вместо того, чтобы отправлять и получать сообщения, имеется ряд встроенных 
функций, которые можно использовать.  Они могут быть вызваны любым процессом, а
не только владельцем порта.

\begin{center}
\begin{tabular}{|>{\raggedright}p{140pt}|>{\raggedright}p{300pt}|}
\hline
\multicolumn{2}{|p{440pt}|}{Встроенные функции для работы с портами} 
	\tabularnewline
\hline
\texttt{port\_command(Порт, Данные)}  & 
Отправляет \texttt{Данные} в \texttt{Порт} \tabularnewline
\hline
\texttt{port\_close(Порт)}  & 
Закрывает \texttt{Порт} \tabularnewline
\hline
\texttt{port\_connect(Порт, НовыйPid)}  & 
Устанавливает владельца \texttt{Порта} равным \texttt{НовомуPid}. Старый 
владелец остаётся связанным с портом и должен сам вызвать \texttt{unlink(Порт)}
если связь не требуется. \tabularnewline
\hline
\texttt{erlang:port\_info(\\
Порт, Элемент)}  & 
Возвращает информацию о \texttt{Порте} с ключом \texttt{Элемент} \tabularnewline
\hline
\texttt{erlang:ports()}  & 
Возвращает список всех открытых портов на текущем узле \tabularnewline
\hline
\end{tabular}
\end{center}

Есть несколько дополнительных встроенных функций, которые применимы только к 
драйверам портов: это \texttt{port\_control/3} и \texttt{erlang:port\_call/3}.