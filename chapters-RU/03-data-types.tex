\chapter{Типы данных (термы)}
\label{datatypes}


\section{Унарные (одиночные) типы данных}


\subsection{Атомы}
\label{datatypes:atom}
\textbf{Атом} --- это символьное имя, также известное, как \emph{литерал}.
Атомы начинаются со строчной латинской буквы и могут содержать буквенно-цифровые
символы, подчёркивания (\texttt{\_}) и символ at (\texttt{@}).
Как вариант, атомы могут быть указаны с помощью заключения в одиночные кавычки
(\texttt{'}), это необходимо, если атом начинается с заглавной буквы, или содержит
другие символы, кроме подчёркиваний и at (\texttt{@}). Например:

\begin{verbatim}
hello
phone_number
'Monday'
'phone number'
\end{verbatim}

\texttt{'Anything inside quotes \textbackslash n\textbackslash 012'}
\hfill(см. раздел \ref{datatypes:escapeseq})


\subsection{Истина и ложь}
\label{datatypes:boolean}
В Erlang нет специального типа данных для \textbf{логических} значений. Эту роль
выполняют атомы \texttt{true} и \texttt{false}.

\texttt{2 =< 3} \resultingin \texttt{true} \\
\texttt{true or false} \resultingin \texttt{true}


\subsection{Целые числа}
\label{datatypes:integer}

В дополнение к обычному способу записи \textbf{целых чисел} Erlang предлагает
ещё ряд способов записи. Запись \texttt{\$Char} даёт числовое значение для 
символа '\texttt{Char}' в кодировке Latin-1 (это также может быть и непечатный
символ) и запись \texttt{Base\#Value} --- даёт целое число, записанное с 
основанием \texttt{Base}, основание должно быть целым числом в диапазоне $2..36$.

\texttt{42} \resultingin \texttt{42} \\
\$A  \resultingin \texttt{65} \\
\texttt{\$\textbackslash n} \resultingin \texttt{10}\hfill(see section \ref{datatypes:escapeseq}) \\
\texttt{2\#101} \resultingin \texttt{5} \\
\texttt{16\#1f} \resultingin \texttt{31}


\subsection{Действительные с плавающей точкой}
\label{datatypes:float}

Действительное число \textbf{с плавающей точкой} записывается, как\linebreak
\texttt{Основание}[\texttt{"e"Экспонента}], где \texttt{Основание} --- это
действительное число в диапазоне от 0.01 до 10000 и \texttt{Экспонента} 
(необязательное) --- это целое число со знаком, указывающее экспоненту (степень
десятки, на которую множится \texttt{Основание}). Например:

\texttt{2.3e-3} \resultingin \texttt{2.30000e-3}\hfill
(соответствует $2.3 \cdot 10^{-3}$)


\subsection{Ссылочные значения}
\label{datatypes:reference}

\textbf{Ссылка} --- это терм, значение которого является уникальным в системе 
времени исполнения Erlang, который создаётся встроенной функцией 
\texttt{make\_ref/0}. (Для дополнительной информации по встроенным функциям,
или \emph{BIF}-ам, смотрите раздел \ref{functions:bifs}.)


\subsection{Порты}
\label{datatypes:port}

\textbf{Идентификатор порта} указывает на открытый в системе порт (смотрите главу
\ref{ports} про порты).


\subsection{Идентификаторы процессов (Pid)}
\label{datatypes:pid}

\textbf{Идентификатор процесса}, или \emph{pid}, указывает на процесс в системе 
(смотрите главу \ref{processes} о процессах).


\subsection{Анонимные функции}
\label{datatypes:fun}

Тип данных \emph{fun} идентифицирует \textbf{функциональный объект} (смотрите
раздел \ref{functions:funs}).



\section{Составные типы данных}


\subsection{Кортежи}
\label{datatypes:tuple}

\textbf{Кортеж} --- это составной тип данных, который содержит \textbf{фиксированное 
количество термов}, заключённое \{в фигурные скобки\}. Например:

\texttt{\{Терм1,...,ТермN\}}

Каждый из \texttt{ТермX} в кортеже называется \textbf{элементом}. Количество 
элементов называется \textbf{размером} данного кортежа.

\begin{center}
\begin{tabular}{|>{\raggedright}p{220pt}|>{\raggedright}p{230pt}|}
\hline
\multicolumn{2}{|p{321pt}|}{BIF-функции для работы с кортежами}\tabularnewline
\hline
\texttt{size(Кортеж)} & 
Возвращает размер \texttt{Кортежа}\tabularnewline
\hline
\texttt{element(N,Кортеж)} &
Возвращает \texttt{N}\textsuperscript{ый} элемент в \texttt{Кортеже} 
\tabularnewline
\hline
\texttt{setelement(N,Кортеж1,Выражение)} &
Возвращает новый кортеж, скопированный из \texttt{Кортеж1}, в котором 
\texttt{N}\textsuperscript{ый} элемент заменён новым значением 
\texttt{Выражение}\tabularnewline
\hline
\end{tabular}
\end{center}

\texttt{P = \{adam, 24, \{july, 29\}\}} \resultingin \texttt{P} связано с \texttt{\{adam, 24, \{july, 29\}\}} \\
\texttt{element(1, P)} \resultingin \texttt{adam} \\
\texttt{element(3, P)} \resultingin \texttt{ \{july,29\}} \\
\texttt{P2 = setelement(2, P, 25)} \resultingin \texttt{P2} связано с \texttt{\{adam, 25, \{july, 29\}\}} \\
\texttt{size(P)} \resultingin \texttt{3} \\
\texttt{size(\{\})} \resultingin \texttt{0} \\


\subsection{Записи}
\label{datatypes:record}

\textbf{Запись} это \emph{именованный кортеж}, имеющий именованные элементы,
которые называются \textbf{полями}. Тип записи определяется в виде атрибута 
модуля, например:

\begin{verbatim}
-record(Запись, {Поле1 [= Значение1],
                 ...
                 ПолеN [= ЗначениеN]}).
\end{verbatim}

Здесь имя записи \texttt{Запись} и имена полей \texttt{ПолеХ} --- атомы и каждое
\texttt{ПолеX} может получить необязательное значение по умолчанию
\texttt{ЗначениеX}. Это определение может быть помещено в любом месте модуля между
определениями функций, но обязательно до первого использования. Если тип записи
используется в нескольких модулях, рекомендуется поместить его в отдельный файл
для включения.

Новая запись с типом \texttt{Запись} создаётся с помощью следующего выражения:

\begin{verbatim}
#Запись{Поле1=Выражение1, ..., ПолеK=ВыражениеK [, _=ВыражениеL]}
\end{verbatim}

Поля не обязательно должны идти в том же порядке, как и в определении записи.
Пропущенные поля получат свои соответствующие значения по умолчанию. Если 
использована последняя запись (подчёркивание равно \texttt{ВыражениеL}), то все
оставшиеся поля получат значение \texttt{ВыражениеL}. Поля без значений по
умолчанию и пропущенные поля получат значение \texttt{undefined}.

Значение поля можно получить используя выражение \texttt{Переменная\#Запись.Поле}.

\begin{erlang}
-module(employee).
-export([new/2]).
-record(person, {name, age, employed=erixon}).

new(Name, Age) -> #person{name=Name, age=Age}.
\end{erlang}

Здесь функция \texttt{employee:new/2} может быть использована в другом модуле,
который также должен включить файл, содержащий определение использованной здесь
записи \texttt{person}.

\texttt{\{P = employee:new(ernie,44)\}} \resultingin \texttt{\{person, ernie, 44,
erixon\}} \\
\texttt{P\#person.age} \resultingin \texttt{44} \\
\texttt{P\#person.employed} \resultingin \texttt{erixon}

При работе с записями в интерпретаторе Erlang, можно использовать функции\linebreak
\texttt{rd(ИмяЗаписи, ОпределениеЗаписи)} и \texttt{rr(Модуль)} для того, чтобы
определить или загрузить новые определения записей. Подробнее читайте в
документации\linebreak\emph{Erlang Reference Manual}.



\subsection{Списки}
\label{datatypes:list}

\textbf{Список} --- это составной тип данных, который содержит \emph{переменное}
количество \textbf{термов}, заключённое в квадратных скобках.

\texttt{[Терм1,...,ТермN]}

Каждый \texttt{ТермX} в списке называется \textbf{элементом}. \textbf{Длина}
списка означает количество элементов. Как это принято в функциональном 
программировании, первый элемент называется \textbf{головой} списка, и остаток
(начиная с 2\textsuperscript{го} элемента и до конца) называется \textbf{хвостом}
списка. Заметьте, что отдельные элементы в списке не должны быть одинакового типа,
хотя часто практикуется (и это, наверное, даже удобно), иметь в списке элементы
одинакового типа --- когда приходится работать с элементами разных типов, обычно
используются \textbf{записи}.

\begin{center}
\begin{tabular}{|>{\raggedright}p{124pt}|>{\raggedright}p{290pt}|}
\hline
\multicolumn{2}{|p{321pt}|}{BIF для работы со списками} \tabularnewline
\hline
\texttt{length(Список)} &
Возвращает длину \texttt{Списка} \tabularnewline
\hline
\texttt{hd(Список)} &
Возвращает 1\textsuperscript{й} элемент \texttt{Списка} (голову) \tabularnewline
\hline
\texttt{tl(Список)} &
Возвращает \texttt{Список} без 1\textsuperscript{го} элемента (хвост) \tabularnewline
\hline
\end{tabular}
\end{center}

Оператор "<вертикальная черта"> (\textbar{}, также в некоторых книгах по ФП он 
называется \texttt{cons}), отделяет ведущие элементы списка (один или более)
от остальных элементов. Например:

\texttt{[H | T]  = [1, 2, 3, 4, 5]} \resultingin \texttt{H=1} и \texttt{T=[2, 3, 4, 5]} \\
\texttt{[X, Y | Z] = [a, b, c, d, e]} \resultingin \texttt{X=a}, \texttt{Y=b} и \texttt{Z=[c, d, e]}

Список --- рекурсивная структура. Неявно список завершается ссылкой на пустой 
список, то есть~\texttt{[a, b]} это то же самое, как и \texttt{[a, b | []]}. 
Список, выглядящий как \texttt{[a, b | c]} называется \textbf{плохо 
сформированным} (badly formed) и такой записи следует избегать (потому что
атом '\texttt{c}', завершающий структуру списка сам \emph{не} является списком).
Списки натурально способствуют рекурсивному функциональному программированию.
Например, следующая функция \texttt{sum} вычисляет сумму списка и функция 
\texttt{double} умножает каждый элемент списка на 2, при этом конструируя и 
возвращая новый список по ходу выполнения.

\begin{erlang}
sum([]) -> 0;
sum([H | T]) -> H + sum(T).

double([]) -> [];
double([H | T]) -> [H*2 | double(T)].
\end{erlang}

Определения функций выше представляют \emph{сопоставление с образцом}, которое
описано далее в главе \ref{patterns}. Образцы в такой записи часто встречаются в 
рекурсивном программировании, неявно предоставляя "<базовый случай"> (для пустого
списка в этих примерах).

Для работы со списками, оператор \texttt{++} соединяет вместе два списка 
(присоединяет второй аргумент к первому) и возвращает список-результат.  Оператор 
\texttt{--} создаёт список, который является копией первого аргумента, за тем
исключением, что для каждого элемента во втором списке его первое вхождение в
результат (если такое было) удаляется.

\texttt{[1,2,3] ++ [4,5]} \resultingin \texttt{[1,2,3,4,5]}
\texttt{[1,2,3,2,1,2] -- [2,1,2]} \resultingin \texttt{[3,1,2]}

Подборка функция, работающих со списками может быть найдена в модуле стандартной
библиотеки под названием \texttt{lists}.


\subsection{Строки}
\label{datatypes:string}

\textbf{Строки} --- это цепочки символов, заключённые между двойными кавычками,
на самом деле они хранятся в памяти, как списки целых чисел --- символов.

\texttt{"abcdefghi"} то же самое, как и \texttt{[97,98,99,100,101,102,103,104,105]}

\texttt{""} то же самое, как и \texttt{[]}

Две строки, записанные подряд без разделительных знаков и операторов будут 
соединены в одну во время компиляции и не принесут дополнительных затрат по
соединению во время исполнения.

\texttt{"string" "42"} \resultingin \texttt{"string42"}


\subsection{Двоичные данные}
\label{datatypes:binary}

Двоичные данные --- это блок нетипизированной памяти, по умолчанию является 
последовательностью 8-битных элементов (байтов).

\texttt{<}\texttt{<Элемент1,...,ЭлементN>}\texttt{>}

Каждый \texttt{ЭлементX} указывается в виде
\texttt{Значение}[\texttt{:Размер}][\texttt{/СписокСпецификацийТипа}].

\begin{center}
\begin{tabular}{|>{\raggedright}p{120pt}|>{\raggedright}p{120pt}|>{\raggedright}p{180pt}|}
\hline
\multicolumn{3}{|p{297pt}|}{Спецификация элемента двоичных данных}\tabularnewline
\hline
\texttt{Значение} &
\texttt{Размер} &
\texttt{СписокСпецификацийТипа}\tabularnewline
\hline
Результат выражения должен быть целым, действительным числом или двоичными
данными. & 
Результат выражения должен быть целым числом & 
Последовательность необязательных спецификаторов типа, в любом порядке, разделённых
дефисами (-)\tabularnewline
\hline
\end{tabular}
\end{center}

% page break at this point, so turned into two chunks.
\begin{center}
\begin{tabular}{|>{\raggedright}p{90pt}|>{\raggedright}p{140pt}|>{\raggedright}p{200pt}|}
\hline
\multicolumn{3}{|p{297pt}|}{Спецификаторы типов}\tabularnewline
\hline
Тип данных &
\texttt{integer} \textbar{} \texttt{float} \textbar{} \texttt{binary} &
По умолчанию \texttt{integer}\tabularnewline
\hline
Наличие знака & 
\texttt{signed} \textbar{} \texttt{unsigned} & 
По умолчанию без знака, \texttt{unsigned}\tabularnewline
\hline
Порядок байтов (endianness) &
\texttt{big} \textbar{} \texttt{little} \textbar{} \texttt{native} &
Зависит от архитектуры процессора. По умолчанию \texttt{big}\tabularnewline
\hline
Единица измерения (бит) & 
\texttt{unit:}\emph{ЦелоеЧисло} & 
Разрешены значения от 1 до 256.
Default is 1 for integer and float, and 8 for binary\tabularnewline
\hline
\end{tabular}
\end{center}

Значение \texttt{Размера} умножается на единицу измерения и даёт число бит, которые
может занять данный сегмент двоичных данных. Каждый сегмент может состоять из нуля
или более битов, но общая сумма битов должна делиться нацело на 8, иначе во время 
исполнения возникнет ошибка \texttt{badarg}. Также сегмент с типом \texttt{binary}
должен иметь размер, делящийся нацело на 8.

Двоичные данные не могут вкладываться друг в друга.

\begin{erlang}
<<1, 17, 42>>       % <<1, 17, 42>>
<<"abc">>           % <<97, 98, 99>> (То же, что и <<$a, $b, $c>>)
<<1, 17, 42:16>>    % <<1,17,0,42>>
<<>>                % <<>>
<<15:8/unit:10>>    % <<0,0,0,0,0,0,0,0,0,15>>
<<(-1)/unsigned>>   % <<255>>
\end{erlang}


\section{Escape-последовательности}
\label{datatypes:escapeseq}
Escape-последовательности разрешено использовать в строках и атомах, которые 
заключены в кавычки.

\begin{center}
\begin{tabular}{|>{\raggedright}p{91pt}|>{\raggedright}p{229pt}|}
\hline
\multicolumn{2}{|p{321pt}|}{Escape-последовательности}\tabularnewline
\hline
\textbackslash{}b & Backspace (удаление слева)\tabularnewline
\hline
\textbackslash{}d & Delete (удаление справа)\tabularnewline
\hline
\textbackslash{}e & Escape\tabularnewline
\hline
\textbackslash{}f & Новая страница (при печати)\tabularnewline
\hline
\textbackslash{}n & Новая строка\tabularnewline
\hline
\textbackslash{}r & Возврат курсора в начало\tabularnewline
\hline
\textbackslash{}s & Пробел\tabularnewline
\hline
\textbackslash{}t & Табуляция\tabularnewline
\hline
\textbackslash{}v & Вертикальная табуляция\tabularnewline
\hline
\textbackslash{}XYZ, \textbackslash{}XY, \textbackslash{}X &
Символ, записанный в восьмеричном представлении, как XYZ, XY или X\tabularnewline
\hline
\textbackslash{}\textasciicircum{}A .. \textbackslash{}\textasciicircum{}Z & 
Комбинации клавиш от Ctrl-A до Ctrl-Z\tabularnewline
\hline
\textbackslash{}\textasciicircum{}a .. \textbackslash{}\textasciicircum{}z &
Комбинации клавиш от Ctrl-A до Ctrl-Z\tabularnewline
\hline
\textbackslash{}' & Единичная кавычка \tabularnewline
\hline
\textbackslash{}\textbf{\texttt{"}} & Двойная кавычка \tabularnewline
\hline
\textbackslash{}\textbackslash{} & Обратная косая черта (\textbackslash) 
\tabularnewline
\hline
\end{tabular}
\end{center}


\section{Преобразования типов}

Есть множество встроенных в стандартную библиотеку функций, предназначенных для
преобразования между типами данных:

\begin{center}
\begin{tabular}{|>{\raggedright}p{63pt}|>{\raggedright}p{30pt}|>{\raggedright}p{35pt}|>{\raggedright}p{21pt}|>{\raggedright}p{21pt}|>{\raggedright}p{21pt}|>{\raggedright}p{30pt}|>{\raggedright}p{21pt}|>{\raggedright}p{35pt}|}
\hline
\multicolumn{9}{|p{243pt}|}{Преобразования типов}\tabularnewline
\hline
 & atom & integer & float & pid & fun & tuple & list & binary\tabularnewline
\hline
atom &  & - & - & - & - & - & X & X\tabularnewline
\hline
integer & - &  & X & - & - & - & X & X\tabularnewline
\hline
float & - & X &  & - & - & - & X & X\tabularnewline
\hline
pid & - & - & - &  & - & - & X & X\tabularnewline
\hline
fun & - & - & - & - &  & - & X & X\tabularnewline
\hline
tuple & - & - & - & - & - &  & X & X\tabularnewline
\hline
list & X & X & X & X & X & X &  & X\tabularnewline
\hline
binary & X & X & X & X & X & X & X & \tabularnewline
\hline
\end{tabular}
\end{center}

Встроенная функция \texttt{float/1} переводит целые числа в числа с 
плавающей точкой. Встроенные функции \texttt{round/1} и \texttt{trunc/1} переводят
числа с плавающей точкой обратно в целые, округляя или отбрасывая дробную часть.

Функции \texttt{Тип\_to\_list/1} и \texttt{list\_to\_Тип/1}
переводят различные типы в списки (строки) и из списков.

Функции \texttt{term\_to\_binary/1} и \texttt{binary\_to\_term/1} переводят любое
значение в закодированную двоичную форму и обратно (Подробнее: 
\url{http://erlang.org/doc/apps/erts/erl_ext_dist.html}).

\begin{erlang}
atom_to_list(hello)        % "hello"
list_to_atom("hello")      % hello
float_to_list(7.0)         % "7.00000000000000000000e+00"
list_to_float("7.000e+00") % 7.00000
integer_to_list(77)        % "77"
list_to_integer("77")      % 77
tuple_to_list({a, b ,c})   % [a,b,c]
list_to_tuple([a, b, c])   % {a,b,c}
pid_to_list(self())        % "<0.25.0>"
term_to_binary(<<17>>)     % <<131,109,0,0,0,1,17>>
term_to_binary({a, b ,c})  % <<131,104,3,100,0,1,97,100,0,1,98,
                           %   100,0,1,99>>
binary_to_term(<<131,104,3,100,0,1,97,100,0,1,98,100,0,1,99>>)  
                           % {a,b,c}
term_to_binary(math:pi())  % <<131,99,51,46,49,52,49,53,57,50,54,
                           % 53,51,...>>
\end{erlang}

