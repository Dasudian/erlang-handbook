\chapter{Обработка ошибок}
\label{errorhandling}

Эта глава описывает обработку ошибок внутри процесса. Такие ошибки известны ещё
под названием \textbf{исключения}.


\section{Классы исключений и причины ошибок}

\begin{center}
\begin{tabular}{|>{\raggedright}p{55pt}|>{\raggedright}p{350pt}|}
\hline
\multicolumn{2}{|p{326pt}|}{Классы исключений}\tabularnewline
\hline
\texttt{error} &
Ошибка времени выполнения, например, при применении оператора к недопустимым 
типам аргументов. Ошибки времени выполнения могут быть вызваны программистом с
помощью встроенной функции \texttt{erlang:error(Причина)} или 
\texttt{erlang:error(Причина, Аргументы)} \tabularnewline
\hline
\texttt{exit}  &
Процесс вызвал \texttt{exit(Причина)}, см. раздел \ref{processes:termination}
о завершении процессов \tabularnewline
\hline
\texttt{throw}  & 
Процесс вызвал \texttt{throw(Выражение)}, см. раздел 
\ref{errorhandling:catchthrow} о бросках исключений\tabularnewline
\hline
\end{tabular}
\end{center}

Появление исключения приводит к аварийной остановке процесса, то есть его 
исполнение останавливается и он и его данные удаляются из системы. Также это 
действие называется \emph{уничтожением} (termination). После этого сигналы 
выхода посылаются всем связанным процессам. Исключение состоит из класса, 
причины выхода и копии стека вызовов. Стек вызовов можно сформировать и 
получить в удобном виде с помощью функции \texttt{erlang:get\_stacktrace/0}.

Ошибки времени исполнения и другие исключения могут не приводить к смерти 
процесса, если использовать выражения \texttt{try} и \texttt{catch}.

Для исключений, имеющих класс \texttt{error}, например для обычных ошибок 
времени выполнения, \textbf{причиной выхода} будет кортеж \texttt{\{Причина, 
Стек\}} где \texttt{Причина} --- это терм, указывающий более точно на тип 
ошибки.

\begin{center}
\begin{tabular}{|>{\raggedright}p{110pt}|>{\raggedright}p{320pt}|}
\hline
\multicolumn{2}{|p{326pt}|}{Причины выхода}\tabularnewline
\hline
\texttt{badarg}  & 
Передан аргумент недопустимого типа. \tabularnewline
\hline
\texttt{badarith}  &
Аргумент в арифметическом выражении имеет недопустимый тип. \tabularnewline
\hline
\texttt{\{badmatch, Значение\}}  & 
Вычисление сопоставления с образцом прошло неудачно. \texttt{Значение} не совпало
с образцом.
\tabularnewline
\hline
\texttt{function\_clause}  & 
Не найдено ни одного подходящего уравнения функции по образцам аргументов и
охранных выражений при вычислении вызова функции. \tabularnewline
\hline
\texttt{\{case\_clause, Значение\}}  & 
При вычислении выражения \texttt{case} \texttt{Значение} не совпало ни с чем. \tabularnewline
\hline
\texttt{if\_clause}  & 
Ни одна из веток выражения \texttt{if} не оказалась равна \texttt{true}.
\tabularnewline
\hline
\texttt{\{try\_clause, Value\}}  &
При вычислении секции выражения \texttt{try} со \texttt{Значением} не совпала ни
одна из веток. \tabularnewline
\hline
\texttt{undef}  & 
Функция не была найдена при попытке исполнить вызов функции \tabularnewline
\hline
\texttt{\{badfun, Fun\}}  & 
Что-то не так с анонимной функцией \texttt{Fun} \tabularnewline
\hline
\texttt{\{badarity, Fun\}}  & 
Функция вызвана с неверным количеством аргументов. Значение \texttt{Fun} описывает
и её и переданные аргументы \tabularnewline
\hline
\texttt{timeout\_value}  & 
Значение таймаута в \texttt{receive}$...$\texttt{after} было вычислено и оказалось
не целым 32-битным числом и не атомом \texttt{infinity} \tabularnewline
\hline
\texttt{noproc}  & 
Попытка создать связь с несуществующим процессом \tabularnewline
\hline
\texttt{\{nocatch, Значение\}}  & 
Попытка выполнить \texttt{throw} за пределами кода, защищённого оператором
\texttt{catch}. \texttt{Значение} --- терм, который был брошен \tabularnewline
\hline
\texttt{system\_limit}  & Сработало одно из системных ограничений, заданных 
реализацией виртуальной машины или операционной системы \tabularnewline
\hline
\end{tabular}
\end{center}

\texttt{Стек} --- это цепочка вызовов функций, которые исполнялись в тот момент,
когда произошла ошибка, даётся в виде списка кортежей \texttt{\{Модуль, Имя, 
Арность\}}, где первым идёт самый недавний вызов. Иногда самый последний вызов 
вместо \texttt{Арности} будет содержать \texttt{Аргументы}: \texttt{\{Модуль, 
Имя, Аргументы\}}.


\section{Catch и throw}
\label{errorhandling:catchthrow}
\begin{erlangru}
catch Выражение
\end{erlangru}

Такая запись возвращает результат вычисления \texttt{Выражения} если во время 
вычисления не возникло исключение. В случае исключения возвращаемое значение 
будет кортежем с информацией о нём.

\begin{erlangru}
{'EXIT', {Причина, Стек}}
\end{erlangru}

Такое исключение считается \emph{пойманным}. Непойманное исключение приведёт к 
уничтожению процесса. Если исключение было вызвано вызовом функции 
\texttt{exit(Терм)} то будет возвращён кортеж с причиной в виде 
\texttt{\{'EXIT',Терм\}}. Если исключение возникло при вызове 
\texttt{throw(Терм)}, тогда будет возвращено значение \texttt{Терма}.

\texttt{catch 1+2} \resultingin \texttt{3}\\
\texttt{catch 1+a } \resultingin \texttt{\{'EXIT',\{badarith,[...]\}\}}

Оператор \texttt{catch} имеет низкий приоритет и выражения, использующие его 
часто требуют заключения в блок \texttt{begin..end} или круглые скобки.

\texttt{A = (catch 1+2)} \resultingin \texttt{3}

Встроенная функция \texttt{throw(Выражение)} используется для 
\emph{нелокального} выхода из функций. Её можно вызывать только под защитой
\texttt{catch}, что возвратит результат вычисления \texttt{Выражения}.

\texttt{catch begin 1,2,3,throw(four),5,6 end} \resultingin \texttt{four}

Если \texttt{throw/1} вычисляется за пределами оператора \texttt{catch}, то 
возникнет ошибка времени выполнения \texttt{nocatch}.

Оператор \texttt{catch} не спасёт процесс от завершения по сигналу выхода из 
другого связанного с ним процесса (если только не был включен режим перехвата 
сигналов выхода, trap\_exit).


\section{Try}
\label{errorhandling:try}

Выражение \texttt{try} способно различить различные классы исключений.  
Следующий пример эмулирует поведение описанного чуть выше \texttt{catch
Выражение}:

\begin{erlang}
try Expr
catch
    throw:Term -> Term;
    exit:Reason -> {'EXIT', Reason};
    error:Reason -> {'EXIT',{Reason, erlang:get_stacktrace()}}
end
\end{erlang}

Полное описание \texttt{try} следующее:

\begin{erlangru}
try Выражение [of
    Образец1 [when ОхранныеВыражения1] -> Тело1;
    ...;
    ОбразецN [when ОхранныеВыраженияN] -> ТелоN]
[catch
    [КлассОшибки1:]ОбразецИсключения1 [when ОхранаИсключения1] ->
        ТелоИсключения1;
    ...;
    [КлассОшибкиN:]ОбразецИсключенияN [when ОхранаИсключенияN] ->
        ТелоИсключенияN]
[after ТелоПосле]
end
\end{erlangru}

В наличии должны обязательно быть как минимум одна строка \texttt{catch} или 
одно предложение \texttt{after}.  Может дополнительно использоваться оператор
\texttt{of} после \texttt{Выражения}, который добавляет вычисление \texttt{case}
к значению \texttt{Выражения}.

\texttt{try} возвратит значение \texttt{Выражения}, если только не произойдёт 
исключение во время его вычисления.  Тогда исключение \emph{ловится} и ряд 
\texttt{ОбразцовИсключения} с подходящим \texttt{КлассомОшибки} сопоставляются
один за другим с пойманным исключением.  Если \texttt{КлассОшибки} не указан, то
подразумевается \texttt{throw}. Если сопоставление удачно проходит и 
необязательные выражения \texttt{ОхраныИсключения} тоже равны \texttt{true}, то
вычисляется соответствующее \texttt{ТелоИсключения} и его результат становится
возвращаемым значением.

Если не найдено подходящего \texttt{ОбразцаИсключения} с таким же 
\texttt{КлассомОшибки} и выражениями \texttt{ОхраныИсключения}, равными 
\texttt{true}, то исключение передаётся дальше, как если бы начальное 
\texttt{Выражение} не было заключено в \texttt{try}.  Исключение, происходящее
во время вычисления \texttt{ТелаИсключения} не будет поймано.

Если ни один из \texttt{Образцов} не совпал, то произойдёт ошибка времени 
исполнения \texttt{try\_clause}.

Если определено, то \texttt{ТелоПосле} всегда вычисляется \textbf{после} всех 
операций в\linebreak
\texttt{try..catch}, независимо от возникновения ошибки. Его 
возвращаемое значение игнорируется и не влияет на значение всего оператора 
\texttt{try} (как если бы \texttt{after} не было). \texttt{ТелоПосле} 
вычисляется даже если исключение произошло в \texttt{Теле} или 
\texttt{ТелеИсключения}, в этом случае исключение передаётся выше по коду.

%% The \texttt{AfterBody} is evaluated after either \texttt{Body} or
%% \texttt{ExceptionBody} no matter which one. The evaluated value of the
%% \texttt{AfterBody} is lost; the return value of the try expression is
%% the same with an after section as without. Even if an exception occurs
%% during evaluation of \texttt{Body} or \texttt{ExceptionBody}, the
%% \texttt{AfterBody} is evaluated. In this case the exception is caught
%% and passed on after the \texttt{AfterBody} has been evaluated, so the
%% exception from the try expression is the same with an after section as
%% without.

Исключение, которое происходит во время вычисления \texttt{ТелаПосле} не 
ловится, то есть если \texttt{ТелоПосле} вычислилось по причине исключения в 
\texttt{Выражении}, \texttt{Теле} или \texttt{ТелеИсключения}, то оригинальное
исключение будет потеряно и вместо него полёт вверх по стеку вызовов продолжит 
уже новое исключение.

