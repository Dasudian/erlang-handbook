\chapter{Структура Erlang-программы}

\section{Синтаксис модулей}

Программа на Erlang состоит из \textbf{модулей} где каждый модуль является текстовым
файлом с расширением \textbf{.erl}. Для небольших программ все модули обычно хранятся
в одной директории. Модуль состоит из атрибутов модуля и определений функций.

\begin{erlang}
-module(demo).
-export([double/1]).

double(X) -> times(X, 2).

times(X, N) -> X * N.
\end{erlang}

Показанный модуль \texttt{demo} состоит из функции \texttt{times/2}, которая является
локальной для модуля и функции \texttt{double/1}, которая экспортирована и может быть
вызвана снаружи модуля.

\texttt{demo:double(10)} \resultingin \texttt{20}\hfill
(стрелка $\Rightarrow$ читается как "<даёт результат">)

\texttt{double/1} означает, что перед нами функция "<double"> с \textit{одним}
аргументом. Функция \texttt{double/2} принимает \textit{два} аргумента и считается
другой,  отличной от первой, функцией. Количество аргументов называется
\textbf{арностью} (arity) данной функции.



\section{Атрибуты модулей}
\textbf{Атрибут модуля} определяет некоторое свойство модуля и состоит из 
\textbf{метки} и \textbf{значения} в следующем виде:

\texttt{-Tag(Value).}

Метка \texttt{Tag} должна быть атомом, в то время, как значение \texttt{Value} 
должно быть термом (смотрите главу \ref{datatypes}). Можно указывать любое имя
для атрибута модуля, также разрешены повторы. Атрибуты хранятся в скомпилированном
коде и могут быть извлечены с помощью вызова функции
\texttt{Module:module\_info(attributes).}

\subsection{Предопределённые атрибуты модулей}
Предопределённые атрибуты модулей должны быть размещены в модуле до начала первого
определения функции.

\begin{itemize}

	\item \verb|-module(Module).|\hfill\\
	Этот атрибут обязателен и должен идти первой строкой файла. Он определяет имя
	модуля. Имя Module, атом (смотрите раздел \ref{datatypes:atom}),
    должно соответствовать имени файла без расширения \texttt{.erl}.

	\item \verb|-export([Func1/Arity1, ..., FuncN/ArityN]).|\hfill\\
	Этот атрибут указывает, какие функции модуля могут быть вызваны снаружи модуля. Каждое имя функции \texttt{Func\textsubscript{X}} --- это атом и арность функции
	\texttt{ArityX} --- целое число.

	\item \verb|-import(Module,[Func1/Arity1, ..., FuncN/ArityN]).|\hfill\\
	Этот атрибут указывает модуль \texttt{Module}, из которого импортируется список
	функций. Например:

	\begin{itemize}
	\item \verb|import(demo, [double/1]).|\hfill\\
	Эта запись означает, что теперь можно писать \texttt{double(10)} вместо более
	длинной записи \texttt{demo:double(10)}, которая может быть непрактична, если
	функция используется часто.
	\end{itemize}

	\item \verb|-compile(Options).|\hfill\\
	Параметры компилятора для данного модуля.

	\item \verb|-vsn(Vsn).|\hfill\\
	Версия модуля. Если не указать этот атрибут, то по умолчанию будет использована
	контрольная сумма содержимого модуля.

	\item \verb|-behaviour(Behaviour).|\hfill\\
	Этот атрибут указывает поведение модуля, либо выбранное пользователем, либо 
	одно из стандартных поведений OTP: \texttt{gen\_server}, \texttt{gen\_fsm},
	\texttt{gen\_event} или \texttt{supervisor}. Принимаются как британская 
	(behaviour), так и американская запись (behavior).
\end{itemize}



\subsection{Определения записей и макросов}

Записи и макросы определяются так же, как и другие атрибуты модуля:

\begin{erlang}
-record(Record,Fields).

-define(Macro,Replacement).
\end{erlang}

Записи и определения макросов также позволяются между функциями, если определение
встречается раньше, чем его первое использование. (Подробнее о записях читайте 
секцию \ref{datatypes:record}, а о макросах главу \ref{macros}.)



\subsection{Включение содержимого файлов}

Включение содержимого файлов указывается аналогично другим атрибутам модуля:

\begin{erlang}
-include(File).

-include_lib(File).
\end{erlang}

\texttt{File} --- это строка, представляющая собой имя файла. Включаемые файлы 
обычно используются для определений записей и макросов, которые используются
в нескольких модулях сразу. По договорённости для включаемых файлов используется
расширение \texttt{.hrl}.

\begin{erlang}
-include("my_records.hrl").
-include("incdir/my_records.hrl").
-include("/home/user/proj/my_records.hrl").
\end{erlang}

Если путь к файлу \texttt{File} начинается с компонента пути \texttt{\$Var}, то
значение переменной окружения \texttt{Var} (которое возвращается функцией
\texttt{os:getenv(Var)}) будет подставлено вместо \texttt{\$Var}.

\begin{erlang}
-include("$PROJ_ROOT/my_records.hrl").
\end{erlang}
%%$ texmaker parser bug

Директива \texttt{include\_lib} подобна \texttt{include}, но считается, что 
первый компонент пути --- имя приложения..

\begin{erlang}
-include_lib("kernel/include/file.hrl").
\end{erlang}

Сервер кода (специальный процесс, управляющий загрузкой и версиями модулей в памяти) 
использует функцию \texttt{code:lib\_dir(kernel)} для поиска директории текущей
(свежайшей) версии модуля \texttt{kernel}, и затем в поддиректори \texttt{include} 
производится поиск нужного файла \texttt{file.hrl}.



\section{Комментарии}

Комментарии могут появляться в любом месте модуля, кроме как внутри строк и атомов,
заключённых в кавычки. Комментарий начинается с символа процента (\texttt{\%}) и
действителен до конца строки но не включая символ конца строки. Завершающий строку
символ конца строки, с точки зрения компилятора, действует как пробел.



\section{Кодировка файлов}

Erlang работает с полным набором символов Latin-1 (ISO-8859-1). Таким образом, 
можно использовать и выводить на экран все печатные символы из набора Latin-1 без
цитирования с помощью обратной косой черты (\textbackslash{}). Атомы и переменные 
могут использовать все символы из набора Latin-1.

\vspace*{12pt}
\begin{center}
\begin{tabular}{|>{\raggedright}p{52pt}|>{\raggedright}p{53pt}|>{\raggedright}p{103pt}|>{\raggedright}p{87pt}|}
\hline
\multicolumn{4}{|p{297pt}|}{Классы символов в кодировке}\tabularnewline
\hline
Восьме\-ричное & Деся\-тичное~ &   & Класс \tabularnewline
\hline
40 -  57 & 32 - 47 &  ! \texttt{"} \# \$ \% \& ' / & Символы пунктуации \tabularnewline
\hline
60 -  71 & 48 - 57 & 0 - 9 & Десятичные цифры \tabularnewline
\hline
72 - 100 & 58 - 64 & : ; \texttt{<} = \texttt{>} @ & Символы пунктуации \tabularnewline
\hline
101 - 132 &  65 - 90 & A - Z & Заглавные буквы \tabularnewline
\hline
133 - 140 &  91 - 96 & [ \textbackslash{} ] \textasciicircum{} \_ ` & Символы пунктуации \tabularnewline
\hline
141 - 172 &  97 - 122 & a  -  z & Строчные буквы \tabularnewline
\hline
173 - 176 & 123 - 126 & \{ \textbar{} \} \textasciitilde{} & Символы пунктуации \tabularnewline
\hline
200 - 237 & 128 - 159 ~ &   & Управляющие символы \tabularnewline
\hline
240 - 277 & 160 - 191 & - ¿  & Символы пунктуации \tabularnewline
\hline
300 - 326 & 192 - 214 & À - Ö  & Заглавные буквы \tabularnewline
\hline
327  & 215 & ×  & Символ пунктуации \tabularnewline
\hline
330 - 336 & 216 - 222 & Ø - Þ  & Заглавные буквы \tabularnewline
\hline
337 - 366 & 223 - 246 & ß - ö  & Строчные буквы \tabularnewline
\hline
367  & 247 & ÷  & Символ пунктуации \tabularnewline
\hline
370 - 377 & 248 - 255 & ø - ÿ  & Строчные буквы \tabularnewline
\hline
\end{tabular}
\end{center}

% because of where this lands, force a page break to avoid orphan.
\newpage
\section{Зарезервированные слова}

%\vspace{12pt}

Следующие ключевые слова в Erlang зарезервированы, и не могут использоваться в
качестве атомов (для использования в качестве атома должны быть заключены в 
одиночные кавычки):

\begin{erlang}
after and andalso band begin bnot bor bsl bsr bxor case catch cond
div end fun if let not of or orelse receive rem try when xor
\end{erlang}
